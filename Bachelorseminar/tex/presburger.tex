\subsection{Kodierung}
\begin{frame}
    \begin{itemize}[<+->]
      \item $\Sigma = \{0,1\}^{n}$, for free variables $x_1, \dotsc x_n$.
      \item $L\left(\varphi\right) = \displaystyle\bigcup_{s \in Sol\left(\varphi\right)} \texttt{LSBF}(s)$
    \end{itemize}
    \uncover<+->{\textbf{Example:}
    $$
      \texttt{LSBF}(5, 10, 0) = \vektora{1}{0}{0}\vektora{0}{1}{0}\vektora{1}{0}{0}\vektora{0}{1}{0}\left(\vektora{0}{0}{0}\right)^*
      \quad \begin{array}{r} 2^0 + 2^2 = 5 \\ 2^1 + 2^3 = 3 \\ 0 = 0 \end{array}
    $$
    }
\end{frame}

\subsection{Formel}
\begin{frame}
\begin{algorithm}[H]
  \SetKwFunction{FtoNFA}{FtoNFA}
  \SetKwFunction{EqtoDFA}{EqtoDFA}
  \SetKwFunction{Free}{Free}
  \SetKwFunction{Projection}{Projection}
  \SetKwFunction{Union}{Union}
  \SetKwFunction{CompNFA}{CompNFA}
  \SetKwFunction{DFAtoNFA}{DFAtoNFA}
  \SetKwProg{function}{Function}{}{}
  \function{\FtoNFA{$\varphi$}}{
    \Switch{$\varphi$}{
      \Case{$\neg \varphi$}{
        \Return \CompNFA{\FtoNFA{$\varphi$}}\;
      }
      \Case{$\varphi_1 \lor \varphi_2$}{
        $I$ $\leftarrow$ \Free{$\varphi$} $\cup$ \Free{$\varphi$}\;
        $s_1$ $\leftarrow$ \Projection{$I$,\FtoNFA{$\varphi_1$}}\;
        $s_2$ $\leftarrow$ \Projection{$I$,\FtoNFA{$\varphi_2$}}\;
        \Return \Union{$s_1, s_2$}\;
      }
      \Case{$\exists x. \varphi$}{
        $I$ $\leftarrow$ \Free{$\varphi$} $-$ $x$\;
        \Return \Projection{$I$,\FtoNFA{$\varphi$}}\;
      }
      \Other{
        \Return \DFAtoNFA{\EqtoDFA{$\varphi$}}\;
      }
    }
  }{}
\end{algorithm}
\end{frame}

\subsection{Gleichungen}
\begin{frame}
  \uncover<+->{Betrachtung von Gleichungen der Form $a_1x_1 + \dotsc + a_nx_n = ax = b$
  \\[2\baselineskip]
  }
  \uncover<+->{\textbf{Idee:}\\
  Jeder Zustand $q \in \mathbb{Z}$ erkennt jene Wörter $c \in \mathbb{N}^n$, für die $ac = q$ gilt.
  }
\end{frame}

\begin{frame}{$q \xrightarrow{\zeta} q'$?}
  \begin{align*}
    \uncover<+->{     & c' \in L(q')\\}
    \uncover<+->{\iff & 2c' + \zeta \in L(q) &\text{(def LSBF)} \\}
    \uncover<+->{\iff & a (2c' + \zeta) = q &\text{(Idee)} \\}
    \uncover<+->{\iff & a c' = \frac{1}{2}(q-a\zeta)}
  \end{align*}
  \uncover<+->{
    \[ \delta(q,\zeta) =
    \begin{cases}
      q_t                   & \quad \text{if } q = q_t \text{ or } q-a\zeta \text{ is odd}\\
      \frac{1}{2}(q-a\zeta) & \quad \text{if } q-a\zeta \text{ is even}\\
    \end{cases}
    \]
  }
\end{frame}

\begin{frame}
\begin{algorithm}[H]
  \SetKwFunction{EqtoDFA}{EqtoDFA}
  \SetKwFunction{pick}{pick}
  \EqtoDFA{$\varphi$}\\
  \KwData{Equation $\varphi ::= ax = b$}
  \KwResult{DFA $A_\varphi = (Q,\Sigma,\delta,q_0,F)$ such that $L(A_\varphi) = L(\varphi)$}
  $Q, \delta, F \leftarrow \emptyset$\;
  $q_0 \leftarrow s_b$\;
  $W \leftarrow \left\{s_b\right\}$\;
  \While{$W \neq \emptyset$}{
    $s_k \leftarrow$ \pick{$W$}\;
    $Q \leftarrow Q \cup \left\{s_k\right\}$\;
    \lIf{$k = 0$}{$F \leftarrow F \cup \left\{s_k\right\}$}
    \ForAll{$\zeta \in \{0,1\}^{n}$}{
      \uIf{$k-a\zeta$ is even}{
        $j \leftarrow \frac{1}{2}(k-a\zeta)$\;
        \lIf{$s_j \notin Q$}{$W \leftarrow W \cup \left\{s_j\right\}$}
        $\delta \leftarrow \delta \cup \left(s_k, \zeta, s_j\right)$\;
      }
    }
  }
\end{algorithm}
\end{frame}

% \begin{frame}
%   \begin{center}
%   \scalebox{.83}{
%     \begin{minipage}{1.2048192771\textwidth}
%     \begin{tikzpicture}[->,>=stealth',node distance=5cm,auto,semithick,bend angle=12]
%       \node[initial,state]   (A) {$2$};
%       \node[state]           (B) [below right of=A, xshift=-1cm, yshift=-.5cm] {$1$};
%       \node[state,accepting]           (C) [right of=A] {$0$};
%       \node[state] (D) [right of=B] {$-1$};
%       \node[state] (E) [right of=C] {$-2$};
%
%       \path (A)
%                 edge [loop above] node {$\vektorb{1}{0}{1}$} (A)
%                 edge [bend left] node [pos=0.3] {$\vektorb{0}{0}{0},\vektorb{1}{1}{1}$} (B)
%                 edge          node {$\vektorb{0}{1}{0}$} (C)
%             (B) edge [bend left] node {$\vektorb{0}{0}{1}$} (A)
%                 edge [loop below] node {$\vektorb{0}{1}{1}$} (B)
%                 edge [bend left] node [pos=0.3] {$\vektorb{1}{0}{0}$} (C)
%                 edge [bend left] node {$\vektorb{1}{1}{0}$} (D)
%             (C) edge [bend left] node {$\vektorb{1}{0}{1}$} (B)
%                 edge [loop above] node {$\vektorb{0}{0}{0},\vektorb{1}{1}{1}$} (C)
%                 edge [bend left] node {$\vektorb{0}{1}{0}$} (D)
%             (D) edge [bend left] node {$\vektorb{0}{0}{1}$} (B)
%                 edge [bend left] node {$\vektorb{0}{1}{1}$} (C)
%                 edge [loop below] node  {$\vektorb{1}{0}{0}$} (D)
%                 edge [bend left] node {$\vektorb{1}{1}{0}$} (E)
%             (E) edge             node [above] {$\vektorb{1}{0}{1}$} (C)
%                 edge [bend left] node {$\vektorb{0}{0}{0},\vektorb{1}{1}{1}$} (D)
%                 edge [loop above] node {$\vektorb{0}{1}{0}$} (E);
%     \end{tikzpicture}
%     \end{minipage}
%   }
%   $\varphi ::= x + 2y - 3z$
%   \end{center}
% \end{frame}

\begin{frame}
  \begin{center}
  \scalebox{.83}{
    \begin{minipage}{1.2048192771\textwidth}
    \begin{tikzpicture}[->,>=stealth',node distance=5cm,auto,semithick,bend angle=12]
      \only<1-3>{\node[initial,state,fill={rgb:red,1;green,2;blue,5}] (A) {$2$};}
      \visible<1->{\path (A) edge [loop above] node {$\vektorb{1}{0}{1}$} (A);}
      \only<2-3>{\node[state] (B) [below right of=A, xshift=-1cm, yshift=-.5cm] {$1$};}
      \only<2->{\path (A) edge [bend left] node [pos=0.3] {$\vektorb{0}{0}{0},\vektorb{1}{1}{1}$} (B);}
      \only<3-7>{\node[state,accepting] (C) [right of=A] {$0$};}
      \only<3->{\path (A) edge node {$\vektorb{0}{1}{0}$} (C);}

      \only<4->{\node[initial,state]   (A) {$2$};}
      \only<4-7>{\node[state] (B) [below right of=A, xshift=-1cm, yshift=-.5cm,fill={rgb:red,1;green,2;blue,5}] {$1$};}
      \only<4->{\path (B) edge [bend left] node {$\vektorb{0}{0}{1}$} (A);}
      \only<5->{\path (B) edge [loop below] node {$\vektorb{0}{1}{1}$} (B);}
      \only<6->{\path (B) edge [bend left] node [pos=0.3] {$\vektorb{1}{0}{0}$} (C);}
      \only<7-10>{\node[state] (D) [right of=B] {$-1$};}
      \only<7->{\path (B) edge [bend left] node {$\vektorb{1}{1}{0}$} (D);}

      \only<8->{\node[state] (B) [below right of=A, xshift=-1cm, yshift=-.5cm] {$1$};}
      \only<8-10>{\node[state,accepting] (C) [right of=A,fill={rgb:red,1;green,2;blue,5}] {$0$};}
      \only<8->{\path (C) edge [bend left] node {$\vektorb{1}{0}{1}$} (B);}
      \only<9->{\path (C) edge [loop above] node {$\vektorb{0}{0}{0},\vektorb{1}{1}{1}$} (C);}
      \only<10->{\path (C) edge [bend left] node {$\vektorb{0}{1}{0}$} (D);}

      \only<11->{\node[state,accepting] (C) [right of=A] {$0$};}
      \only<11-14>{\node[state] (D) [right of=B,fill={rgb:red,1;green,2;blue,5}] {$-1$};}
      \only<11->{\path (D) edge [bend left] node {$\vektorb{0}{0}{1}$} (B);}
      \only<12->{\path (D) edge [bend left] node {$\vektorb{0}{1}{1}$} (C);}
      \only<13->{\path (D) edge [loop below] node  {$\vektorb{1}{0}{0}$} (D);}
      \only<14-14>{\node[state] (E) [right of=C] {$-2$};}
      \only<14->{\path (D) edge [bend left] node {$\vektorb{1}{1}{0}$} (E);}

      \only<15->{\node[state] (D) [right of=B] {$-1$};}
      \only<15-17>{\node[state] (E) [right of=C,fill={rgb:red,1;green,2;blue,5}] {$-2$};}
      \only<15->{\path (E) edge node [above] {$\vektorb{1}{0}{1}$} (C);}
      \only<16->{\path (E) edge [bend left] node {$\vektorb{0}{0}{0},\vektorb{1}{1}{1}$} (D);}
      \only<17->{\path (E) edge [loop above] node {$\vektorb{0}{1}{0}$} (E);}

      \only<18>{\node[state] (E) [right of=C] {$-2$};}
    \end{tikzpicture}
    \end{minipage}
  }
  $\varphi ::= x + 2y - 3z = 2$
  \end{center}
\end{frame}


\begin{frame}
  \begin{Lemma}
    Sei $\varphi ::= ax=b$ und $s := \sum^k_{i=1}|a_i|$ Für jeden durch \texttt{EqtoDFA}$(\varphi)$ erzeugten Zustand $j$ gilt, $-|b| - s \leq j \leq |b| + s$.
  \end{Lemma}
  \begin{Beweis}
    Induktion über die Anzahl der erzeugten Zustände.\\
    Basisfall: Ist für $s_b$ erfüllt.\\
    Schritt: Der neue Zustand $s_j$ wird durch $s_k$ erreicht.
    $$
    -|b| - s \leq \frac{-|b| - s -a\zeta}{2} \leq j \leq \frac{|b| + s -a\zeta}{2} \leq \frac{|b| + 2s}{2} \leq |b| + s
    $$
  \end{Beweis}
  \begin{Korollar}
    \texttt{EqtoDFA}$(\varphi)$ terminert für jedes $\varphi$.
  \end{Korollar}
\end{frame}

\begin{frame}
  \begin{Lemma}
    Sei $\varphi ::= ax=b$ und $A_\varphi$ der von \texttt{EqtoDFA}$(\varphi)$ erzeugte Automat.\\
    $\forall q \in Q. \forall w \in \left(\left\{0,1\right\}^n\right)^*.\left(w \in L(q) \iff a\texttt{LSBF}^{-1}(w) = q\right)$
  \end{Lemma}
  \begin{Beweis}
    Induktion über die Anzahl der erzeugten Zustände.\\
    Basisfall: Defintion der Endzustände.\\
    Schritt: IH + Konstruktionseigenschaft.
  \end{Beweis}
  \begin{Korollar}
    \texttt{EqtoDFA}$(\varphi)$ ist total korrekt.
  \end{Korollar}
\end{frame}

\subsection{Ungleichungen}
\begin{frame}[fragile]
  \begin{grammar}
    <formula> ::= \lit{$\neg$} <formula> \alt <formula> \lit{$\lor$} <formula> \alt \lit{$\exists$} <var> <formula> \alt <atomicformula>

    <atomicformula> ::= <term> \lit{$=$} <term> \alt <term> \lit{$\leq$} <term>

    <term> ::= <term> \lit{$+$} <term> \alt <variable> \alt <constant>

    <variable> ::= x | y | z | ...

    <constant> ::= 0 | 1
  \end{grammar}
\end{frame}

\begin{frame}
\begin{algorithm}[H]
  \SetKwFunction{IqtoDFA}{IqtoDFA}
  \SetKwFunction{pick}{pick}
  \IqtoDFA{$\varphi$}\\
  \KwData{Inequation $\varphi ::= ax \leq b$}
  \KwResult{DFA $A_\varphi = (Q,\Sigma,\delta,q_0,F)$ such that $L(A_\varphi) = L(\varphi)$}
  $Q, \delta, F \leftarrow \emptyset$\;
  $q_0 \leftarrow s_b$\;
  $W \leftarrow \left\{s_b\right\}$\;
  \While{$W \neq \emptyset$}{
    $s_k \leftarrow$ \pick{$W$}\;
    $Q \leftarrow Q \cup \left\{s_k\right\}$\;
    \lIf{$k \geq 0$}{$F \leftarrow F \cup \left\{s_k\right\}$}
    \ForAll{$\zeta \in \{0,1\}^{n}$}{
      $j \leftarrow \left\lfloor \frac{1}{2}(k-a\zeta) \right\rfloor$\;
      \lIf{$s_j \notin Q$}{$W \leftarrow W \cup \left\{s_j\right\}$}
      $\delta \leftarrow \delta \cup \left(s_k, \zeta, s_j\right)$\;
    }
  }
\end{algorithm}
\end{frame}
