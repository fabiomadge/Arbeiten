\documentclass{article}
\usepackage[utf8]{inputenc}

\begin{document}

\subsection{General Remarks}
\begin{itemize}
    \item Make it clear earlier, that the focus is a reorganized version of the base type, using a zipper. Access is only cheap for specific data.
    \item Why did the Zipper become a Context?
    \item \texttt{[a] @ suffix = a : suffix}
    \item process of taking a derivative from a real\emph{?} function.
    \item \emph{datatype} 'a list
    \item "A magical expression" is not a sensible subsection of "Context of Algebraic Datatypes"
    \item Try to motivate introducing combinatorial spacies first. Then actually do it.
    \item Sources would have been helpful    
\end{itemize}

\subsection{Grammar/Spelling}
\begin{itemize}
    \item are basic components \emph{to} most functional programming
    \item \emph{A} Zipper is a data structure that helps \emph{manipulating} \emph{a} functional datatype.
    \item The operations being applied \emph{on} this tree include\emph{s}
    \item The operations on list\_focus include\emph{s} moving it to one ste
    \item Algebraic datatype follow\emph{s} many common algebraic laws.
    \item Two subtleties \emph{lie} here
    \item the inductively defined species \emph{is} equivalent
    \item two species with this \emph{egf} have the same number
    \item insignificant case names \emph{of} whatever follows
    \item For example\emph{,} the species defined
    \item We can also define the derivative \emph{for} a species.
    \item where the right\emph{-}hand side also
    \item The idea is similar \emph{as} when we define the context of algebraic datatypes.
    \item the inductively defined species is equivalent \emph{with} a possibly infinite sum of
\end{itemize}
\end{document}