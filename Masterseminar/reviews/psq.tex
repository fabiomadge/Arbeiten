\documentclass{article}
\usepackage[utf8]{inputenc}

\begin{document}

\subsection{General Remarks}
\begin{itemize}
    \item Meta: Was hat es mit den Punkten und Kommata in den Definitionen auf sich? Warum ist die Verwendung nicht konsistent?
    \item Meta: Mehr Spalten führen nur zu unnötigen Worttrennungen.
    \item 2.1: Zieh die Definition vom Datentyp nach vorne. Mir wurde erst damit klar was es mit den zusätzlichen Keys auf sich hat.
    \item 2.2: Verletzen die eckigen Klammern nicht deine Key condition?
    \item 2.2.2: delMin Void = emptyPSQ suggeriert mehr Komplexität als eigentlich vorhanden
    \item 3: Hättest du am ehesten weglassen können.
    \item 4: comprehensible implementation for the implementation of a priority search queue
    \item 4.1: expression q of type q :: PSQ k v
    \item 4.2: heavily -> intensively 
    \item 4.3: “potentially” und  “if the key exists” scheinen redundant
    \item 4.3: Wie kannst du Dinge mergen, die vorher nicht separat existiert haben?
    \item 4.3 “which might support” Unter welchen Umständen?
    \item 5: “might possibly”
    \item 5.1: Proportional, oder einfach nur die Höhe?    
\end{itemize}

\subsection{Grammar/Spelling}
\begin{itemize}
    \item original code to this new syntax\emph{,} and to provide additional
    \item lookup, insertion, update\emph{,} and deletion of bindings
    \item For the remaining invariants\emph{,} it is convenient
    \item search tree
    \item key, as otherwise\emph{,} the operator would
    \item pattern matching
    \item Equipped with the basic implementation \emph{from} Section 2
    \item Next, the cost \emph{for} the operations
    \item perform this operation for a list of vertices \emph{as} once
    \item which is not \emph{require} here
    \item priority search queue has \emph{lead} to an efficient implementation
    \item approach \emph{on} the implementation
\end{itemize}
\end{document}