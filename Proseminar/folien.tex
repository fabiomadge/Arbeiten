\documentclass[]{beamer}
\usepackage[utf8]{inputenc}
\usepackage[ngerman]{babel}
\usepackage[german, vlined, ruled]{algorithm2e}
\usepackage{graphics}
\usepackage{etoolbox}
\usetheme{Berlin}
\setbeamertemplate{footline}{}
\setbeamertemplate{headline}{}
\setbeamertemplate{theorems}[ams style]
\usecolortheme{dove}
\usefonttheme{professionalfonts}
\beamertemplatenavigationsymbolsempty

\undef{\Theorem}
\undef{\Lemma}
\undef{\Definition}
\newtheorem{Proposition}{Proposition}
\newtheorem{Theorem}{Theorem}
\newtheorem{Lemma}{Lemma}
\newtheorem{Definition}{Definition}

\begin{document}

\title{Eine Lösung des Stable Marriage Problems und drei Erweiterungen}
\author[F. Madge Pimentel]{Fabio Madge Pimentel}
\institute{Technische Universität München}
\date[13.01.16]{13. Januar 2016}
\maketitle

\begin{frame}
\setcounter{tocdepth}{1}
\tableofcontents[pausesections]
\end{frame}

\AtBeginSection[]
{
\begin{frame}
\setcounter{tocdepth}{2}
\tableofcontents[currentsection,hideothersubsections]
\end{frame}
}

\AtBeginSubsection[]
{
\begin{frame}
\setcounter{tocdepth}{2}
\tableofcontents[currentsection,
        currentsubsection,
        subsectionstyle=show/shaded/hide]
\end{frame}
}

\section{Grundlagen}
\subsection{Allgemein}
\begin{frame}
  \begin{itemize}[<+->]
  	\item Matchingproblem im vollständigen bipartiten Graph $G = (V,E)$
    \item $V = M \cup W$
    \item $M \cap W = \emptyset$
    \item $|M| = |W|$
    \item Präferenzlisten - totale Ordnung der anderen Teilmenge
    \item Ziel: \textit{stabile} Paarung $P \subseteq M \times W$
  \end{itemize}
\end{frame}

\begin{frame}
  \begin{Definition}
  \label{stabil}
    Eine Paarung $P$ ist genau dann nicht \textit{stabil}, wenn es zwei Knoten $m \in M$ und $w \in W$ gibt, für die gilt, dass $p_{P}(w) <_{w} m$ und $p_{P}(m) <_{m} w$.
  \end{Definition}
\end{frame}

\subsection{Problembeschreibung}
\begin{frame}
  \begin{itemize}[<+->]
  	\item Männer und Frauen einander gerecht zuteilen
    \item Anzahl der Männer und Frauen gleich
    \item Ordnung aller Personen des anderen Geschlechts, nach Zuneigung
    \item Kein blockierendes Paar
  \end{itemize}
\end{frame}

\begin{frame}
  \begin{algorithm}[H]
  \SetKwData{KwM}{m}\SetKwData{KwW}{w}
  \FuerAlle{$\KwM \in M$}{
    \FuerJedes{$\KwW \in W . \KwW <_{\KwM} p_{P}(\KwM)$}{
      \Wenn{$\KwM <_{\KwW} p_{P}(\KwW)$}{
        \Return{instabile Paarung}\;
      }
    }
  }
  \Return{stabile Paarung}\;
  \caption{Stabilitätsüberprüfung}
\end{algorithm}

\end{frame}

\subsection{Gale–Shapley Algorithmus}
\begin{frame}
  \begin{algorithm}[H]
  \SetKwData{KwM}{m}\SetKwData{KwM'}{m'}\SetKwData{KwW}{w}
  setze alle Personen auf $frei$\;
  \Solange{ein Mann \KwM keine Partnerin hat}{
    \KwW $\leftarrow$ höchstpriorisierte Frau, um deren Hand \KwM nicht angehalten hat\;
    \eWenn{\KwW keinen Partner hat}{
      setze \KwW und \KwM als jeweiligen Partner\;
    }
    {
      \eWenn{\KwW \KwM ihrem Verlobten \KwM' vorzieht}{
        setze \KwW und \KwW als jeweiligen Partner und \KwM' auf $frei$\;
      }{
        \KwW lehnt den Antrag ab\;
      }
    }
  }
  \caption{Gale–Shapley Algorithmus}
\end{algorithm}

\end{frame}

\begin{frame}
  \begin{Lemma}
  \label{gsa_terminiert}
    Für jedes Instanz des Stable Marriage Problems terminiert der Gale–Shapley Algorithmus.
  \end{Lemma}

  \begin{Beweis}
  \label{gsa_terminiert_bew}
    \begin{enumerate}[<+->]
      \item Algorithmus terminiert $\Longleftrightarrow$ alle Männer verlobt (Def. Alg.)
      \item Eine Frau kann nur ablehnen, wenn sie verlobt ist. (Def. Alg.)
      \item Ist eine Frau verlobt, wird sie nicht mehr frei. (Def. Alg.)
      \item Annahme: $\exists m \in M. m$ wird von allen Frauen abgelehnt
      \begin{itemize}
          \item[$\Longrightarrow$] Alle Frauen sind verlobt (2)
          \item[$\Longrightarrow$] $|W| < |M|$
          \item Widerspruch (Def. Prob.)
       \end{itemize}
    \end{enumerate}
  \end{Beweis}
\end{frame}

\begin{frame}
  \begin{Theorem}
  \label{paarung_existiert}
    Für jede Instanz des Stable Marriage Problems existiert mindestens eine stabile Paarung. Sie wird vom Gale–Shapley Algorithmus gefunden.
  \end{Theorem}

  \begin{Beweis}
  \label{paarung_existiert_bew}
    \begin{enumerate} [<+->]
        \item Nach Terminierung: vollständige Paarung $P$ (Def. Alg.)
        \item Annahme: $P$ instabil
        \begin{itemize}
            \item[$\Longrightarrow$] $\exists m \in M,w \in W. (m,w) \textrm{ blockiert } P$ (Def. stabil)
            \item[$\Longrightarrow$] $p_{P}(m) <_{m} w$ (Def. stabil)
            \item[$\Longrightarrow$] $m$ hat $w$ einen Antrag gemacht (Def. Alg.), aber
            \begin{itemize}
                \item direkt abgewiesen, oder
                \item später durch $m <_{w} m'$ ersetzt.
            \end{itemize}
            \item[$\Longrightarrow$] $m <_{w} m' \leq_{w} p_{P}(w)$
            \item Widerspruch (Def. stabil)
        \end{itemize}
        \item (Lemma~\ref{gsa_terminiert}) + 1 + 2 $\Longrightarrow$ (Theorem~\ref{paarung_existiert})
     \end{enumerate}
  \end{Beweis}
\end{frame}

\begin{frame}
  \begin{Definition}
  \label{man_optimal}
    Eine Paarung heißt genau dann \textit{man-optimal}, wenn jeder Mann jene Frau zur Partnerin hat, die vom ihm am höchsten priorisiert ist und in irgendeiner stabilen Paarung mit ihm verheiratet ist.
  \end{Definition}

  \begin{Definition}
  \label{woman_pessimal}
    Eine Paarung heißt genau dann \textit{woman-pessimal}, wenn jede Frau jenen Mann zum Partner hat, der vom ihr am niedrigsten priorisiert ist und in irgendeiner stabilen Paarung mit ihm verheiratet ist.
  \end{Definition}
\end{frame}

\begin{frame}
  \begin{Lemma}
  \label{man_optimal_woman_pessimal}
    Ist eine Paarung man-optimal, so muss sie gleichzeitig auch woman-pessimal sein.
  \end{Lemma}

  \begin{Beweis}
  \label{man_optimal_woman_pessimal_bew}
    \begin{itemize} [<+->]
        \item Sei $P$ eine man-optimal Paarung.
        \item Sei $P'$ eine stabile Paarung mit $\exists w \in W.m' = p_{P'}(w) <_{w} m = p_{P}(w)$.
        \item $(m,w)$ blockieren $P'$
        \begin{itemize}
            \item $m' <_{w} m$ ($P'$)
            \item $p_{P'}(m) <_{m} w$ (man-optimal)
         \end{itemize}
        \item Widerspruch (Assm.)
     \end{itemize}
  \end{Beweis}
\end{frame}

\begin{frame}
  \begin{Theorem}
  \label{mann_optimal}
    Der Gale–Shapley Algorithmus ist deterministisch und vom ihm erzeugte Paarungen sind man-optimal und woman-pessimal.
  \end{Theorem}

  \begin{Beweis}
  \label{mann_optimal_bew}
    % Sie $P$ eine vom Algorithmus in beliebiger Ausführungsreihenfolge berechnete Paarung und $P'$ eine stabile Paarung in der $w = p_{P}(m) <_{m} w' = p_{P'}(m)$ gilt. Dies ist gleichbedeutend damit, dass $m$ während der Ausführung durch $m' \in M.m <_{w'} m'$, als Partner von $w'$ ersetzt wurde. Nun wird angenommen, dass dies das erste Vorkommen einer Abweisung eines stabilen Partners war. Dies schwächt den Beweis nicht ab, erlaubt aber die Feststellung, dass $P'$ durch $(m',w')$ blockiert wird, da $m <_{w'} m'$ ($m'$) und $p_{P'}(m') <_{m'} w'$ (erste Ablehnung). $P'$ kann also nicht stabil sein und somit ist der Algorithmus deterministisch. Da jeder Mann seine bestmögliche Partnerin hat($\nexists P'$), ist $P$ man-optimal und somit auch woman-pessimal (Lemma~\ref{man_optimal_woman_pessimal}).
    \begin{itemize} [<+->]
        \item Sei $P$ Ergebnis vom GSA in beliebiger Ausführungsreihenfolge
        \item Sei $P'$ eine stabile Paarung mit $\exists m \in M.w = p_{P}(m) <_{m} w' = p_{P'}(m)$.
        \item[$\Longrightarrow$] Antrag von $m$ an $w'$ durch $m' \in M$ gehindert (Def. Alg.)
        \item Annahme (\OE): erste Abweisung eines stabilen Partners
        \item[$\Longrightarrow$] $p_{P'}(m') <_{m'} w'$ und $m <_{w'} m'$ (hinderung)
        \item[$\Longrightarrow$] $(m',w')$ blockiert $P'$
        \item[$\Longrightarrow$] $P'$ existiert nicht
        \item[$\Longrightarrow$] $P$ ist man-optimal und woman-pessimal (Lemma~\ref{man_optimal_woman_pessimal})
        \item[$\Longrightarrow$] $P$ ist deterministisch
     \end{itemize}
  \end{Beweis}
\end{frame}


\section{Erweiterungen}
\subsection{Mengen unterschiedlicher Größe}
\begin{frame}
  \begin{Definition}
  \label{stabil_diff}
    Eine Paarung $P$ ist genau dann nicht \textit{stabil}, wenn es zwei Knoten $m \in M$ und $w \in W$ gibt für die gilt, dass $p_{P}(w) <_{w} m$ und $p_{P}(m) <_{m} w$. Liefert $p_{P}(x)$ keinen Partner, ist dieses Ergebnis kleiner als jedes Element der Präferenzliste von $x$.
  \end{Definition}
\end{frame}

\begin{frame}
  \begin{Theorem}
  \label{paarung_existiert_diff}
    Für jede Instanz des Stable Marriage Problems, in der $|W| \neq |M|$ gilt, existiert mindestens eine stabile Paarung mit $\min\{|W|,|M|\}$ Paaren. Der Gale–Shapley Algorithmus findet sie.
  \end{Theorem}

  \begin{Beweis}
  \label{paarung_existiert_diff_bew}
    Wie Lemma \ref{paarung_existiert} mit den lokalen Änderungen. Terminierung ist durch die neue Abbruchbedingung gegeben.
  \end{Beweis}
\end{frame}

\begin{frame}
  \begin{Definition}
  \label{vorziehen}
    Seien P und P' Paarungen und eine Person $x$ \textit{bevorzugt} $P$ gegenüber $P'$, wenn gilt $p_{P'}(x) <_{x} p_{P}(x)$. Die Ordnung verhält sich wie in Definition \ref{stabil_diff} für nicht existente Partner. Hat $x$ in beiden Paarungen keinen Partner, ist $x$ indifferent.
  \end{Definition}
\end{frame}

\begin{frame}
  \begin{Lemma}
  \label{strikte_ordnung}
    Seien $P$ und $P'$ stabile Paarungen in denen $p_{P}(m) = w \land p_{P}(w) = m \land p_{P'}(m) \neq w$ gilt. Daraus folgt, dass entweder $m$ $P$ und $w$ $P'$ bevorzugt, oder umgekehrt.
  \end{Lemma}

  \begin{Beweis}
  \label{strikte_ordnung_bew}
    \begin{enumerate}[<+->]
        \item Seien $\mathcal{X} \subseteq M$ und $\mathcal{Y} \subseteq W$ jene Personen die $P$ bevorzugen
        \item Seien $\mathcal{X'} \subseteq M$ und $\mathcal{Y'} \subseteq W$ jene Personen die $P'$ bevorzugen
        \item $\nexists m \in \mathcal{X},w \in \mathcal{Y}. (m,w) \textrm{ in } P$ (stabil)
        \item $\nexists m \in \mathcal{X},w \in W - \mathcal{Y} - \mathcal{Y'}. (m,w) \textrm{ in } P$ (gleicher Partner)
        \item 3 + 4 $\Longrightarrow$ Wenn m Partner in $P$ hat, dann aus $\mathcal{Y'}$
        \item[$\Longrightarrow$] $|\mathcal{X}| \leq |\mathcal{Y'}|$
        \item Analog: $|\mathcal{X'}| \leq |\mathcal{Y}|$
        \item $|\mathcal{X}| + |\mathcal{X'}| = |\mathcal{Y}| + |\mathcal{Y'}|$ (Fakt, alle Partnerwechsler)
        \item[$\Longrightarrow$] $|\mathcal{X}| = |\mathcal{Y'}|\land |\mathcal{X'}| = |\mathcal{Y}|$
    \end{enumerate}
  \end{Beweis}
\end{frame}

\begin{frame}
  \begin{Theorem}
  \label{partner_oder_nicht_diff}
    Wenn $|W| \neq |M|$ gilt, dann $\max\{W,M\} = Q \cup Q' \land  Q \cap Q' = \emptyset$, wobei die Personen in $Q$ in allen stabilen Paarung einen Partner haben, die in $Q'$ in keiner.
  \end{Theorem}

  \begin{Beweis}
  \label{partner_oder_nicht_diff_bew}
    \begin{itemize}[<+->]
      \item Seien $P$ und $P'$ stabile Paarungen
      \item Darstellung in einem gerichteten bipartiten Graph $G$
      \item Verlobungen in $P$ als Kanten $M \rightarrow W$, $P'$ dual.
      \item $\forall p \in V. d^{+}_{G}(p) \leq 1 \land d^{-}_{G}(p) \leq 1$
      \item Sei $x \in \max\{W,M\}$ eine Person die $P$ gegenüber $P'$ vorzieht.
      \item Kreisfreier Pfad: $x,..,y \in \max\{W,M\}$, $d^{+}_{G}(x) = d^{-}_{G}(y) = 0$
      \item Traviersieren des Pfad mit (Lemma \ref{strikte_ordnung}) $\Longrightarrow$ $y$ präferiert $P$
      \item Widerspruch
    \end{itemize}
  \end{Beweis}
\end{frame}

\subsection{Unvollständige Präferenzlisten}
\begin{frame}
  \begin{Theorem}
  \label{partner_oder_nicht}
    Wenn es unvollständige Präferenzlisten gibt, dann $W \cup M = Q \cup Q' \land  Q \cap Q' = \emptyset$, wobei die Personen in $Q$ in allen stabilen Paarung einen Partner haben, die in $Q'$ in keiner.
  \end{Theorem}

  \begin{Beweis}
  \label{partner_oder_nicht_bew}
    \begin{itemize}[<+->]
      \item Seien $P$ und $P'$ stabile Paarungen
      \item Darstellung in einem gerichteten bipartiten Graph $G$
      \item Verlobungen in $P$ als Kanten $M \rightarrow W$, $P'$ dual.
      \item $\forall p \in V. d^{+}_{G}(p) \leq 1 \land d^{-}_{G}(p) \leq 1$
      \item Sei $m \in M$ ein Mann der $P$ gegenüber $P'$ vorzieht.
      \item Kreisfreier Pfad: $x,..,y \in W \cup M$, $d^{+}_{G}(x) = d^{-}_{G}(y) = 0$
      \begin{itemize}
        \item $y \in W \Longrightarrow$ $y$ bevorzugt $P$
        \item $y \in M \Longrightarrow$ $y$ bevorzugt $P'$
      \end{itemize}
      \item Traviersieren des Pfad mit (Lemma \ref{strikte_ordnung})
      \item Widerspruch
    \end{itemize}
  \end{Beweis}
\end{frame}

\begin{frame}
  \begin{Theorem}
  \label{element_enfuegen}
    Wenn ein Mann $m$ in einer Instanz eine Frau $w$ an seine Prä\-fe\-renz\-lis\-te anhängt, dann gilt weder in der man-optimal noch in der woman-optimal stabilen Paarung für die neue Instanz, dass irgendeine Frau die entsprechende Paarung für die ursprüngliche Instanz bevorzugt und irgendein Mann, bis auf $m$, die entsprechende Paarung für die neue Instanz bevorzugt.
  \end{Theorem}

  % \begin{Beweis}
  % \label{element_enfuegen_bew}
    % Wenn $m$ bereits in der man-optimal Paarung für die ursprüngliche Instanz eine Partnerin hatte, ändert sich nichts, weil $m$ nicht die Gelegenheit bekommt ihr einen Heiratsantrag zu machen. Falls $m$ keine Partnerin hatte, können die Frauen nur profitieren, da sie nur Änderungen akzeptieren die zu einer Verbesserung führen, und die Männer können nur verlieren, indem eine Frau eine Verlobung wieder löst. $m$ bekommt aber die Chance um die Hand von $w$ anzuhalten.\\
    % Wenn $w$ bereits in der woman-optimal Paarung für die ursprüngliche Instanz einen Partner hatte, ändert sich nichts, weil $m$ nicht die Gelegenheit bekommt ihr einen Heiratsantrag zu machen. Hatte $w$ keinen Partner, ergibt sich die Gelegenheit $m$ den Hof zu machen. Dies wird zwar angenommen, wenn $m$ noch keine Partnerin hat, kann aber keine andere Frau benachteiligen, weil jede potentielle Geschädigte weiter vorne in der Prioritätenliste von $m$ steht, und $w$ somit immer ersetzen würde.
  %     \begin{itemize}%[<+->]
  %       \item man-optimal:
  %         \begin{itemize}
  %           \item m hatte bereits eine Partnerin: Keine Veränderung
  %           \item m hatte noch keine Partnerin:
  %           \begin{itemize}
  %             \item $W$: können sich nur verbessern (Def. Alg.)
  %             \item $m$ kann einen zusätzlichen Antrag machen
  %             \item $M \setminus \{m\}$: Gefahr durch Ersetzung mit $m$
  %           \end{itemize}
  %         \end{itemize}
  %       \item woman-optimal:
  %       \begin{itemize}
  %         \item w hatte bereits einen Partner: Keine Veränderung
  %         \item w hatte noch keinen Partner:
  %         \begin{itemize}
  %           \item $W$: können sich nur verbessern (Def. Alg.)
  %           \item $m$ kann einen zusätzlichen Antrag machen
  %           \item $M \setminus \{m\}$: Gefahr durch Ersetzung mit $m$
  %         \end{itemize}
  %       \end{itemize}
  %     \end{itemize}
  % \end{Beweis}
\end{frame}

\subsection{Streng schwach geordnete Präferenzlisten}
\begin{frame}
  \begin{Definition}
  \label{super-stabil}
    Eine Paarung $P$ ist genau dann nicht \textit{super-stabil}, wenn es zwei Knoten $m \in M$ und $w \in W$ gibt für die gilt, dass $m \leq_{w} p_{P}(w), m \neq p_{P}(w)$ und $w \leq_{m} p_{P}(m), w \neq p_{P}(m)$.
  \end{Definition}
\end{frame}

\begin{frame}
  \begin{Proposition}
  \label{keine_super-stabil}
    Für eine beliebige Instanz des SMP mit streng schwach geordnete Präferenzlisten, kann auch keine super-stabile Paarung existieren.
  \end{Proposition}

  \begin{Beweis}
  \label{keine_super-stabil_bew}
  \begin{itemize} [<+->]
      \item Gegenbeispiel: Instanz mit $\forall x y. x \leq_{y} x, x \in \textrm{Pr\"aferenzliste}(y)$
      \item $\forall m \in M,w \in W. (m,w) \textrm{ blockiert jede Paarung}$
   \end{itemize}
  \end{Beweis}
\end{frame}

\begin{frame}
  \begin{Definition}
  \label{streng_stabil}
    Eine Paarung $P$ ist genau dann nicht \textit{streng stabil}, wenn es zwei Knoten $m \in M$ und $w \in W$ gibt für die gilt, dass $m \leq_{w} m' = p_{P}(w), m \neq m'$, $w \leq_{m} w' = p_{P}(m), w \neq w'$ und entweder $m <_{w} m'$, oder $w <_{m} w'$ gilt.
  \end{Definition}
\end{frame}

\begin{frame}
  \scalebox{.87}{
    \begin{minipage}{1.149425287\textwidth}
      \begin{algorithm}[H]
  \SetKwData{KwM}{m}\SetKwData{KwM'}{m'}\SetKwData{KwW}{w}
  setze alle Personen auf $frei$\;
  \Solange{$\forall m \in M.\exists w \in W.$ $m$ darf $w$ einen Heiratsantrag machen}{
    \Wenn{ein Mann \KwM keine Partnerin hat}{
      \FuerAlle{Frauen \KwW, die für \KwM höchste Priorität haben und um deren Hand \KwM nicht anhalten hat}{
        \eWenn{\KwW keinen Partner hat}{
          füge \KwW und \KwM zu den jeweiligen Partnerlisten hinzu\;
        }
        {
          \Wenn{\KwW \KwM ihren Verlobten vorzieht, oder indifferent ist}{
            \Wenn{\KwW \KwM ihren Verlobten vorzieht}{
              entferne \KwW aus allen Partnerlisten\;
              leere die Partnerliste von \KwW\;
            }
            füge \KwW und \KwM zu den jeweiligen Partnerlisten hinzu\;
          }
          \lSonst{lehnt \KwW den Antrag ab}
        }
      }
    }
    \lSonst{wende Taktik an}
  }
  \caption{Angepasster Gale–Shapley Algorithmus}
\end{algorithm}

    \end{minipage}
  }
\end{frame}

\begin{frame}
  \frametitle{Taktik: super-stabil}
  \begin{itemize}[<+->]
    \item Alle Frauen mit mehreren Verlobten, lösen ihre Verlobungen.
    \item Diese Frauen dürfen nur noch Verlobungen höherer Priorität annehmen.
    \item Haben alle Personen nach Terminierung einen Partner, ist die Paarung super-stabil.
  \end{itemize}
\end{frame}

\begin{frame}
  \frametitle{Taktik: streng stabil}
  \begin{itemize}[<+->]
    \item Darstellung aller Personen als Knoten in einem ungerichteter bipartiten Graph
    \item Verlobungsbeziehungen als Kanten
    \item Ein stabiles Matching in diesem Graph ist eine stabile Paarung $\longrightarrow$ der Algorithmus terminiert
    \item Sonst, $\exists \mathcal{M} \subseteq M. |\mathcal{M}| < |\mathcal{W} = \{w | \exists m, w.(m \in \mathcal{M} \land w \in W \land  w \textrm{ ist mit } m \textrm{ verlobt})\}|$
    \item Alle Frauen $\in \mathcal{W}$ mit mehreren Verlobten, lösen ihre Verlobungen.
    \item Diese Frauen dürfen nur noch Verlobungen höherer Priorität annehmen.
    \item Haben alle Personen nach Terminierung einen Partner, ist die Paarung streng stabil.
  \end{itemize}
\end{frame}

\begin{frame}
  \frametitle{Ursprüngliche Definition}
  \begin{itemize}[<+->]
    \item Umwandlung der Präferenzliste zu einer totalen Ordnung
    \item Standardversion vom GSA
    \item Ergebnis ist abhängig von der Umwandlungsstrategie, nicht mehr zwingend man-optimal
  \end{itemize}
\end{frame}


\section{Asymptotische Optimalität}
\begin{frame}
  \begin{center}
    Existiert ein Algorithmus $\in o(n^2)$, der eine stabile Paarung findet, oder sie verifiziert?
  \end{center}
\end{frame}

\subsection{Adversary Arguments}
\begin{frame}
\begin{figure}
    \centering
    \def\svgwidth{0.69\columnwidth}
    \input{./graphics/Battleship_game_board.pdf_tex}
\end{figure}
\end{frame}


\begin{frame}
\begin{itemize}[<+->]
	\item Beweistechnik für untere Schranken
	\item Bestimmung der Eingabe zur Laufzeit
	\item Ziel: Hinauszögern einer definitiven Entscheidung
\end{itemize}
\end{frame}

\subsection{Kanonischen Frauenpräferenz}
\begin{frame}
  \begin{Lemma}
  \label{eine_paarung}
    Für eine Instanz in der alle Frauen die gleiche Präferenzliste haben, gibt es nur eine stabile Paarung.
  \end{Lemma}

  \begin{Beweis}
  \label{eine_paarung_bew}
    % $P$ sei die einzige stabile Paarung, somit auch die woman-pessimal, $P'$ eine beliebige andere. $\mathcal{W} \subseteq W$ enthält alle Frauen die unterschiedliche Partner in den Paarungen haben. $w \in \mathcal{W}$ wird so gewählt, dass $p_{P}(w)$ maximal in $<_{w}$ ist. Jetzt blockieren $(m = p_{P}(w),w)$ aber $P'$, weil $w$ einen schlechteren Partner bekommen hat und $m$ die höchste Priorität unter allen möglichen Partnern hat.
    \begin{itemize}[<+->]
    	\item Sei $P$ die einzige stabile Paarung (man-optimal)
    	\item Sei $P'$ eine beliebige andere Paarung
    	\item Seien $\mathcal{W} \subseteq W$ die Frauen mit unterschiedlichen Partnern
      \item Sei $w \in \mathcal{W}. m = p_{P}(w)$ maximal in $<_{w}$
        \begin{itemize}
          \item[$\Longrightarrow$] $p_{P'}(w)$ $<_{w} m$ (schlechterer Partner)
          \item[$\Longrightarrow$] $p_{P'}(m)$ $<_{m} w$ (man-optimal)
        \end{itemize}
      \item[$\Longrightarrow$] $(m,w)$ blockiert $P'$
      \item[$\Longrightarrow$] $P'$ existiert nicht
    \end{itemize}
  \end{Beweis}
\end{frame}

\begin{frame}
  \begin{Definition}
  \label{kanonische_listen}
    In der \textit{kanonischen Frauenpräferenz} haben alle Frauen die gleiche aufsteigend sortierte Präferenzliste.
  \end{Definition}
\end{frame}

\begin{frame}
  \begin{Lemma}
  \label{beste_frau}
    In einer Instanz die eine kanonischen Frauenpräferenz hat und dadurch nur eine Paarung $P$, hat jeder Mann $i.i<n$ jene höchstpriorisierte Frau zur Partnerin, die keinen Mann $j.j<i$ zum Partner hat.
  \end{Lemma}

  \begin{Beweis}
  \label{beste_frau_bew}
    \begin{itemize}[<+->]
      \item Sei $x \in W.p_{P}(i) <_{i} x \land i < j$
      \item[$\Longrightarrow$] $(i,x)$ blockiert $P$
    \end{itemize}
  \end{Beweis}
\end{frame}

\subsection{Laufzeit: quadratisch}
\begin{frame}
  \begin{Theorem}
  \label{keine_gute_loesung}
    Im schlimmsten Fall muss ein Algorithmus, der Stabilität verifiziert, oder eine stabile Paarung bestimmt, mindestens $n(n-1)/2$ mal auf die Präferenzlisten der Männer zugreifen.
  \end{Theorem}

  % \begin{Beweis}
  % \label{keine_gute_loesung_bew}
  %   Sei $i$ die bestimmende Zahl eines Mannes $m$ und $k$ die Anzahl der bereits bekannten Positionen auf der Präferenzliste von $m$. Solange $k \leq i-1$ gilt, antwortet die Strategie immer mit $k$ auf eine unbekannte Position. Sobald dies nicht mehr gilt, wird die unspezifizierte Position mit der höchsten Priorität auf die Frau $w$ mit der Nummer $i$ gesetzt. Die anderen Positionen können beliebig gesetzt werden. Dies hat zur Folge, dass alle Frauen mit höherer Priorität als der von $w$, eine kleinere Zahl als $w$ haben. Somit stellt Lemma \ref{beste_frau} sicher, dass es nur eine Paarung geben kann und diese ausschließlich aus Identitäten besteht.\\
  %   Solange aber nicht für jeden Mann $i. i < n$ die Frau mit dieser Zahl positioniert wurde, kann der Gegner immer noch seine Strategie ändern und einer Frau mit einem größeren Index eine höhere Priorität in der Liste von Mann $i$ geben. Die Paarung dieser Instanz hat nun aber nach Lemma \ref{beste_frau} eine andere Gestalt. Erst wenn $n(n-1)/2$ Positionen bekannt sind, kann der Gegner seine Strategie nicht mehr ändern.
  % \end{Beweis}
\end{frame}


\end{document}
