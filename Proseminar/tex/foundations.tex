\subsection{Problembeschreibung}

Im Kontext des Szenarios heißt das also, dass es eine Gemeinschaft von Frauen und Männern gibt die alle eine heterosexuelle Ehe eingehen wollen. Dabei muss die Anzahl der Männer gleich der der Frauen sein. Zum Zweck der optimal Verteilung der Heiratswilligen aufeinander, muss jedes Individuum alle Personen des anderen Geschlecht in eine Ordnung bringen. Die erste Person ist jede welche das Individuum am ehsten heiraten möchte. Auf dieser Grundlage soll nun eine stabile Paarung gefunden werden, es wird also jeder Frau ein Mann zugeordet und umgekehrt.

Sabilität heißt hier also, dass es kein Paar aus einer Frau und einem Mann gibt, das nicht in der Paarung verheiratet ist, das sich aber gegenseitig ihrem Ehepartner in der Paarung vorziehen. Ein Algorithmus zur Überprüfung dieser Eigenschaft lässt sich leicht aus Definition \ref{stabil} ableiten.

\begin{algorithm}[H]
  \SetKwData{KwM}{m}\SetKwData{KwW}{w}
  \FuerAlle{$\KwM \in M$}{
    \FuerJedes{$\KwW \in W . \KwW <_{\KwM} p_{P}(\KwM)$}{
      \Wenn{$\KwM <_{\KwW} p_{P}(\KwW)$}{
        \Return{instabile Paarung}\;
      }
    }
  }
  \Return{stabile Paarung}\;
  \caption{Stabilitätsüberprüfung}
\end{algorithm}


\subsection{Gale–Shapley Algorithmus}

Gale und Shapley haben in ihrem Artikel \cite{Gale:1962} nicht nur das Problem vorgestellt, sondern zeitgleich eine Lösung in Form des nach ihnen benannten Alogrithmus präsentiert. Er bildet zusammen mit Lemma \ref{gsa_terminiert} den konstrukiven Beweis von Theorem \ref{paarung_existiert}.

\begin{algorithm}[H]
  \SetKwData{KwM}{m}\SetKwData{KwM'}{m'}\SetKwData{KwW}{w}
  setze alle Personen auf $frei$\;
  \Solange{ein Mann \KwM keine Partnerin hat}{
    \KwW $\leftarrow$ höchstpriorisierte Frau, um deren Hand \KwM nicht angehalten hat\;
    \eWenn{\KwW keinen Partner hat}{
      setze \KwW und \KwM als jeweiligen Partner\;
    }
    {
      \eWenn{\KwW \KwM ihrem Verlobten \KwM' vorzieht}{
        setze \KwW und \KwW als jeweiligen Partner und \KwM' auf $frei$\;
      }{
        \KwW lehnt den Antrag ab\;
      }
    }
  }
  \caption{Gale–Shapley Algorithmus}
\end{algorithm}


\begin{Lemma}
\label{gsa_terminiert}
  Für jedes Instanz des Stable Marriage Problems terminiert der Gale–Shapley Algorithmus.
\end{Lemma}

\begin{Beweis}
\label{gsa_terminiert_bew}
  Der Algorithmus terminiert genau dann, wenn alle Männer verlobt sind. Dieser Fall tritt immer ein, weil kein Mann von allen Frauen abgewiesen werden kann. Denn Frauen können Männer nur abweisen, wenn sie bereits verlobt sind. Damit der Mann vor allen Frauen auf seiner Liste abgelehnt werden kann, müssten sie also alle verlobt sein, das kann aber nicht sein, weil es dafür $n$ verlobte Männer und einen weiteren bräuchte.
\end{Beweis}

\begin{Theorem}
\label{paarung_existiert}
  Für jedes Instanz des Stable Marriage Problems existiert mindestens eine stabile Paarung. Sie wird vom Gale–Shapley Algorithmus gefunden.
\end{Theorem}

\begin{Beweis}
\label{paarung_existiert_bew}
  Die Paare die nach Terminierung des Algorithmus vorliegen sind offensichtlich eine Paarung $P$. Damit $P$ nicht stabil ist, muss sie ein blockierendes Paar $(m,w)$ enthalten. Wenn $w <_{m} p_{P}(m)$ gilt, hat $m$ $w$ zwinged einen Antrag gemacht und wurde entweder direkt abgewiesen, oder anschließend durch $m' <_{w} m$ ersetzt. Daraus folgt $p_{P}(w) <_{w} m$. $(m,w)$ kann also kein blockierendes Paar sein. Nachdem der Algorithmus für jede Eingabe terminiert (Lemma~\ref{gsa_terminiert}), lieft er immer eine stabile Paarung.
\end{Beweis}

Theorem \ref{paarung_existiert} deutet schon an, dass nicht immer nur ein mögliche stabile Paarung existiert. Tatsächlich gibt es in den meisten Fällen eine Vielzahl, die sich in der Menge der stabilen Paarungen befindet. Deshalb beleuchtet Theorem \ref{mann_optimal} die Natur der vom Gale–Shapley Algorithmus berechneten Paarung. Dabei lässt sich auch beobachten, dass, obwohl die Reihenfolge der Heiratsanträge nicht sehr stark geordnet ist, der Algorithmus für jede mögliche Reihenfolge das selbe Ergebnis liefert.

\begin{Definition}
\label{man_optimal}
  Eine Paarung heißt genau dann man-optimal, wenn jeder Mann jene Frau zur Partnerin hat, die vom ihn am höchsten priorisiert ist und in irgendeiner stabilen Paarung mit ihm verheiratet ist.
\end{Definition}

\begin{Definition}
\label{woman_pessimal}
  Eine Paarung heißt genau dann woman-pessimal, wenn jede Frau jenen Mann zum Partner hat, der vom ihr am niedrigsten priorisiert ist und in irgendeiner stabilen Paarung mit ihm verheiratet ist.
\end{Definition}

\begin{Lemma}
\label{man_optimal_woman_pessimal}
  Ist eine Paarung man-optimal, so muss sie gleichzeitig auch woman-pessimal sein.
\end{Lemma}

\begin{Beweis}
\label{man_optimal_woman_pessimal_bew}
  Sei $P$ eine man-optimal Paarung und $P'$ eine stabile Paarung für die $m = p_{P}(w) <_{w} m' = p_{P'}(w)$ gilt. In diesem Fall müssten $(m,w)$ aber $P'$ blockieren, da $m <_{w} m'$ ($P'$) und $w <_{m} p_{P'}(m)$ (man-optimal).
\end{Beweis}

\begin{Theorem}
\label{mann_optimal}
  Der Gale–Shapley Algorithmus ist deterministisch und vom ihm erzeugte Paarungen sind man-optimal und woman-pessimal.
\end{Theorem}

\begin{Beweis}
\label{mann_optimal_bew}
  Sie $P$ eine vom Algorithmus in beliebiger Ausführungsreihenfolge berechnete Paarung und $P'$ eine stabile Paarung in der $w' = p_{P'}(m) <_{m} w = p_{P}(m)$ gilt. Dies ist gleichbedeutend damit, dass $m$ während der Ausführung durch $m'$, für den gilt $m' <_{w'} m$ gilt, als Partner von $w'$ ersetzt wurde. Nun wird angenommen, dass dies das erste Vorkommen einer Abweisung eines stabilen Partners war. Dies schwächt den Beweis nicht ab, erlaubt aber die Feststellung, dass $P'$ durch $(m',w')$ blockiert wird, da $m' <_{w'} m$ [$m'$] und $w' <_{m'} p_{P'}(m')$ [erste Ablehnung]. $P'$ kann also nicht stabil sein und mit der Tatsache, dass Präferenzlisten total geordnet sind, ist der Algorithmus deterministisch. Da jeder Mann seine bestmögliche Partnerin hat($\nexists P'$), ist $P$ man-optimal und somit auch woman-pessimal (Lemma~\ref{man_optimal_woman_pessimal}).
\end{Beweis}
