\subsection{Allgemein}
\begin{frame}
  \begin{itemize}[<+->]
  	\item Matchingproblem im vollständigen bipartiten Graph $G = (V,E)$
    \item $V = M \cup W$
    \item $M \cap W = \emptyset$
    \item $|M| = |W|$
    \item Präferenzlisten - totale Ordnung der anderen Teilmenge
    \item Ziel: \textit{stabile} Paarung $P \subseteq M \times W$
  \end{itemize}
\end{frame}

\begin{frame}
  \begin{Definition}
  \label{stabil}
    Eine Paarung $P$ ist genau dann nicht \textit{stabil}, wenn es zwei Knoten $m \in M$ und $w \in W$ gibt, für die gilt, dass $p_{P}(w) <_{w} m$ und $p_{P}(m) <_{m} w$.
  \end{Definition}
\end{frame}

\subsection{Problembeschreibung}
\begin{frame}
  \begin{itemize}[<+->]
  	\item Männer und Frauen einander gerecht zuteilen
    \item Anzahl der Männer und Frauen gleich
    \item Ordnung aller Personen des anderen Geschlechts, nach Zuneigung
    \item Kein blockierendes Paar
  \end{itemize}
\end{frame}

\begin{frame}
  \begin{algorithm}[H]
  \SetKwData{KwM}{m}\SetKwData{KwW}{w}
  \FuerAlle{$\KwM \in M$}{
    \FuerJedes{$\KwW \in W . \KwW <_{\KwM} p_{P}(\KwM)$}{
      \Wenn{$\KwM <_{\KwW} p_{P}(\KwW)$}{
        \Return{instabile Paarung}\;
      }
    }
  }
  \Return{stabile Paarung}\;
  \caption{Stabilitätsüberprüfung}
\end{algorithm}

\end{frame}

\subsection{Gale–Shapley Algorithmus}
\begin{frame}
  \begin{algorithm}[H]
  \SetKwData{KwM}{m}\SetKwData{KwM'}{m'}\SetKwData{KwW}{w}
  setze alle Personen auf $frei$\;
  \Solange{ein Mann \KwM keine Partnerin hat}{
    \KwW $\leftarrow$ höchstpriorisierte Frau, um deren Hand \KwM nicht angehalten hat\;
    \eWenn{\KwW keinen Partner hat}{
      setze \KwW und \KwM als jeweiligen Partner\;
    }
    {
      \eWenn{\KwW \KwM ihrem Verlobten \KwM' vorzieht}{
        setze \KwW und \KwW als jeweiligen Partner und \KwM' auf $frei$\;
      }{
        \KwW lehnt den Antrag ab\;
      }
    }
  }
  \caption{Gale–Shapley Algorithmus}
\end{algorithm}

\end{frame}

\begin{frame}
  \begin{Lemma}
  \label{gsa_terminiert}
    Für jedes Instanz des Stable Marriage Problems terminiert der Gale–Shapley Algorithmus.
  \end{Lemma}

  \begin{Beweis}
  \label{gsa_terminiert_bew}
    \begin{enumerate}[<+->]
      \item Algorithmus terminiert $\Longleftrightarrow$ alle Männer verlobt (Def. Alg.)
      \item Eine Frau kann nur ablehnen, wenn sie verlobt ist. (Def. Alg.)
      \item Ist eine Frau verlobt, wird sie nicht mehr frei. (Def. Alg.)
      \item Annahme: $\exists m \in M. m$ wird von allen Frauen abgelehnt
      \begin{itemize}
          \item[$\Longrightarrow$] Alle Frauen sind verlobt (2)
          \item[$\Longrightarrow$] $|W| < |M|$
          \item Widerspruch (Def. Prob.)
       \end{itemize}
    \end{enumerate}
  \end{Beweis}
\end{frame}

\begin{frame}
  \begin{Theorem}
  \label{paarung_existiert}
    Für jede Instanz des Stable Marriage Problems existiert mindestens eine stabile Paarung. Sie wird vom Gale–Shapley Algorithmus gefunden.
  \end{Theorem}

  \begin{Beweis}
  \label{paarung_existiert_bew}
    \begin{enumerate} [<+->]
        \item Nach Terminierung: vollständige Paarung $P$ (Def. Alg.)
        \item Annahme: $P$ instabil
        \begin{itemize}
            \item[$\Longrightarrow$] $\exists m \in M,w \in W. (m,w) \textrm{ blockiert } P$ (Def. stabil)
            \item[$\Longrightarrow$] $p_{P}(m) <_{m} w$ (Def. stabil)
            \item[$\Longrightarrow$] $m$ hat $w$ einen Antrag gemacht (Def. Alg.), aber
            \begin{itemize}
                \item direkt abgewiesen, oder
                \item später durch $m <_{w} m'$ ersetzt.
            \end{itemize}
            \item[$\Longrightarrow$] $m <_{w} m' \leq_{w} p_{P}(w)$
            \item Widerspruch (Def. stabil)
        \end{itemize}
        \item (Lemma~\ref{gsa_terminiert}) + 1 + 2 $\Longrightarrow$ (Theorem~\ref{paarung_existiert})
     \end{enumerate}
  \end{Beweis}
\end{frame}

\begin{frame}
  \begin{Definition}
  \label{man_optimal}
    Eine Paarung heißt genau dann \textit{man-optimal}, wenn jeder Mann jene Frau zur Partnerin hat, die vom ihm am höchsten priorisiert ist und in irgendeiner stabilen Paarung mit ihm verheiratet ist.
  \end{Definition}

  \begin{Definition}
  \label{woman_pessimal}
    Eine Paarung heißt genau dann \textit{woman-pessimal}, wenn jede Frau jenen Mann zum Partner hat, der vom ihr am niedrigsten priorisiert ist und in irgendeiner stabilen Paarung mit ihm verheiratet ist.
  \end{Definition}
\end{frame}

\begin{frame}
  \begin{Lemma}
  \label{man_optimal_woman_pessimal}
    Ist eine Paarung man-optimal, so muss sie gleichzeitig auch woman-pessimal sein.
  \end{Lemma}

  \begin{Beweis}
  \label{man_optimal_woman_pessimal_bew}
    \begin{itemize} [<+->]
        \item Sei $P$ eine man-optimal Paarung.
        \item Sei $P'$ eine stabile Paarung mit $\exists w \in W.m' = p_{P'}(w) <_{w} m = p_{P}(w)$.
        \item $(m,w)$ blockieren $P'$
        \begin{itemize}
            \item $m' <_{w} m$ ($P'$)
            \item $p_{P'}(m) <_{m} w$ (man-optimal)
         \end{itemize}
        \item Widerspruch (Assm.)
     \end{itemize}
  \end{Beweis}
\end{frame}

\begin{frame}
  \begin{Theorem}
  \label{mann_optimal}
    Der Gale–Shapley Algorithmus ist deterministisch und vom ihm erzeugte Paarungen sind man-optimal und woman-pessimal.
  \end{Theorem}

  \begin{Beweis}
  \label{mann_optimal_bew}
    % Sie $P$ eine vom Algorithmus in beliebiger Ausführungsreihenfolge berechnete Paarung und $P'$ eine stabile Paarung in der $w = p_{P}(m) <_{m} w' = p_{P'}(m)$ gilt. Dies ist gleichbedeutend damit, dass $m$ während der Ausführung durch $m' \in M.m <_{w'} m'$, als Partner von $w'$ ersetzt wurde. Nun wird angenommen, dass dies das erste Vorkommen einer Abweisung eines stabilen Partners war. Dies schwächt den Beweis nicht ab, erlaubt aber die Feststellung, dass $P'$ durch $(m',w')$ blockiert wird, da $m <_{w'} m'$ ($m'$) und $p_{P'}(m') <_{m'} w'$ (erste Ablehnung). $P'$ kann also nicht stabil sein und somit ist der Algorithmus deterministisch. Da jeder Mann seine bestmögliche Partnerin hat($\nexists P'$), ist $P$ man-optimal und somit auch woman-pessimal (Lemma~\ref{man_optimal_woman_pessimal}).
    \begin{itemize} [<+->]
        \item Sei $P$ Ergebnis vom GSA in beliebiger Ausführungsreihenfolge
        \item Sei $P'$ eine stabile Paarung mit $\exists m \in M.w = p_{P}(m) <_{m} w' = p_{P'}(m)$.
        \item[$\Longrightarrow$] Antrag von $m$ an $w'$ durch $m' \in M$ gehindert (Def. Alg.)
        \item Annahme (\OE): erste Abweisung eines stabilen Partners
        \item[$\Longrightarrow$] $p_{P'}(m') <_{m'} w'$ und $m <_{w'} m'$ (hinderung)
        \item[$\Longrightarrow$] $(m',w')$ blockiert $P'$
        \item[$\Longrightarrow$] $P'$ existiert nicht
        \item[$\Longrightarrow$] $P$ ist man-optimal und woman-pessimal (Lemma~\ref{man_optimal_woman_pessimal})
        \item[$\Longrightarrow$] $P$ ist deterministisch
     \end{itemize}
  \end{Beweis}
\end{frame}
