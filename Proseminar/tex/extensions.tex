Häufig lassen sich reale Fragestellungen nicht genau durch das SMP modellieren, deshalb macht es Sinn  zu betrachten, wie sich das Aufweichen einzelner Kriterien auf das Ergebnis des GSA und die Menge der stabilen Paarungen auswirkt.

\subsection{Mengen unterschiedlicher Größe}
\label{subsec:size}

Hier wird angenommen, dass die Anzahl der Männer und Frauen nicht gleich ist. Dadurch wird eine Anpassung der Definition von Stabilität nötig.

\begin{Definition}
\label{stabil_diff}
  Eine Paarung $P$ ist genau dann nicht \textit{stabil}, wenn es zwei Knoten $m \in M$ und $w \in W$ gibt für die gilt, dass $m <_{w} p_{P}(w)$ und $w <_{m} p_{P}(m)$. Liefert $p_{P}(x)$ keinen Partner, ist das Ergebnis größer als jedes Element der Präferenzliste von $x$.
\end{Definition}

Falls $|W| < |M|$ darf der GSA erst Abbrechen, sobald für alle unverlobten Männer keine Frau existiert, um deren Hand sie anhalten können.

\begin{Theorem}
\label{paarung_existiert_diff}
  Für jede Instanz des Stable Marriage Problems, in der $|W| \neq |M|$ gilt, existiert mindestens eine stabile Paarung mit $\min\{|W|,|M|\}$ Paaren. Der Gale–Shapley Algorithmus findet sie.
\end{Theorem}

\begin{Beweis}
\label{paarung_existiert_diff_bew}
  Wie Lemma \ref{paarung_existiert} mit den lokalen Änderungen. Terminierung ist durch die neue Abbruchbedingung gegeben.
\end{Beweis}

\begin{Definition}
\label{vorziehen}
  Seien P und P' Paarungen und eine Person $x$ \textit{bevorzugt} $P$ gegenüber $P'$, wenn gilt $p_{P}(x) <_{x} p_{P'}(x)$. Die Ordnung verhält sich wie in Definition \ref{stabil_diff} für nicht existente Partner. Hat $x$ in beiden Paarungen keinen Partner, ist $x$ indifferent.
\end{Definition}

\begin{Lemma}
\label{strikte_ordnung}
  Seien $P$ und $P'$ stabile Paarungen für die $p_{P}(m) = w \land p_{P}(w) = m \land p_{P'}(m) \neq w$ gilt. Daraus folgt, dass entweder $m$, oder $w$ $P$ bevorzugt und die andere Person $P'$ bevorzugt.
\end{Lemma}

\begin{Beweis}
\label{strikte_ordnung_bew}
  $\mathcal{X} \subseteq M$ und $\mathcal{Y} \subseteq W$ seien die Mengen von Männern und Frauen, die $P$ gegenüber $P'$ bevorzugen. $\mathcal{X'}$ und $\mathcal{Y'}$ sind analog definiert, wobei $P$ und $P'$ vertauscht sind.\\
  In $P$ kann es das Paar $(m \in \mathcal{X},w \in \mathcal{Y})$ nicht geben, weil $P'$ stabil ist und $(m,w)$ $P$ folglich blockieren würde. Deshalb können die Männer in $\mathcal{X}$ nur mit Frauen aus $\mathcal{Y'}$ in $P$ eine Partnerschaft eingehen, die indifferenten Frauen kommen nicht in Frage, weil sie den Partner nicht wechseln, sie also auch die Partnerin in $P'$ sein müssten. Somit gilt $|\mathcal{X}| \leq |\mathcal{Y'}|$.\\
  Analog dazu gibt es in $P'$ kein Paar $(m \in \mathcal{X'},w \in \mathcal{Y'})$. Hier gilt $|\mathcal{X'}| \leq |\mathcal{Y}|$.\\
  Es muss aber $|\mathcal{X}| + |\mathcal{X'}| = |\mathcal{Y}| + |\mathcal{Y'}|$ gelten, da beide Seiten die Anzahl derer beschreibt, die in $P$ einen anderen Partner als in $P'$ haben. Daraus folgt $|\mathcal{X}| = |\mathcal{Y'}|$ und $|\mathcal{X'}| = |\mathcal{Y}|$. Somit muss jeder Mann in $\mathcal{X}$ eine Partnerin aus $\mathcal{Y'}$ in $P$ haben und jeder Mann in $\mathcal{X'}$ eine Partnerin aus $\mathcal{Y}$ haben.
\end{Beweis}

\begin{Theorem}
\label{partner_oder_nicht_diff}
  Für jede Instanz des Stable Marriage Problems, in der $|W| \neq |M|$ gilt, dass die größere Menge in zwei Untermengen aufgeteilt werden kann, eine mit jenen Personen die in jeder stabilen Paarung einen Partner haben und die andere mit jenen die in keiner einen Partner haben.
\end{Theorem}

\begin{Beweis}
\label{partner_oder_nicht_diff_bew}
  Zwei Paarungen $P$ und $P'$ werden als gerichteter bipartiter Graph mit $\{M,W\}$ als Bipartition dargestellt. Jeder Knoten hat also höchstens den Eingangsgrad eins und den Ausgangsgrad eins.\\
  Sei $x \in \max\{W,M\}$ eine Person die einen Partner in $P$ hat, aber nicht in $P'$, dadurch also $P$ bevorzugt. Somit ist $x$ eine Quelle, von der ein kreisfreier Pfad zu einer Senke $y \in \max\{W,M\}$ ausgeht. Die Senke muss mit Theorem \ref{paarung_existiert_diff} das gleiche Geschlecht wie $x$ haben, weil alle Personen des anderen Geschlechts Partner in beiden Paarungen haben. $y$ bevorzugt offensichtlich $P'$.\\
  Mit Lemma \ref{strikte_ordnung} kann der Graph entlang des Pfandes traversiert werden. Da $x$ $P$ bevorzugt, muss der direkte Nachfolger $P'$ bevorzugen. Wenn dieses Lemma bis zur Senke angewendet wird, ergibt sich dort ein Widerspruch.
\end{Beweis}

\subsection{Unvollständige Präferenzlisten}

Hier wird den Personen erlaubt Präferenzlisten zu erstellen die nicht alle Individuen des anderen Geschlechts enthalten. Die Anderen kommen als Partner nicht in Frage. Auch in diesem Fall findet Definition \ref{stabil_diff} ihre Anwendung.

\begin{Theorem}
\label{partner_oder_nicht}
  Für eine Instanz die unvollständige Präferenzlisten enthält gilt, dass sowohl die Männer, als auch die Frauen in jeweils in zwei Untermengen aufgeteilt werden können, eine mit jenen Personen, die in jeder stabilen Paarung einen Partner haben und die andere mit jenen die in keiner einen Partner haben.
\end{Theorem}

\begin{Beweis}
\label{partner_oder_nicht_bew}
  Hier werden zwei Paarungen $P$ und $P'$ als gerichteter bipartiter Graph mit $\{M,W\}$ als Bipartition dargestellt. Die Kante $\{m,w\}$ existiert, falls $m = p_{P}(w)$ und $\{w,m\}$ existiert, falls $w = p_{P'}(m)$. Jeder Knoten hat also höchstens den Eingangsgrad eins und den Ausgangsgrad eins.\\
  Sei $m$ ein Mann der einen Partner in $P$ hat, aber nicht in $P'$, also $P$ bevorzugt. Somit ist $m$ eine Quelle, von der ein kreisfreier Pfad zu einer Senke ausgeht. Diese ist entweder eine Frau, die folglich $P$ bevorzugt, oder ein Mann, der analog $P'$ präferiert.\\
  Mit Lemma \ref{strikte_ordnung} kann der Graph entlang des Pfandes traversiert werden. Da $m$ $P$ bevorzugt, muss der direkte Nachfolger $P'$ bevorzugen. Wird dieses Lemma bis zur Senke angewendet, ergibt sich dort mit der obigen Beobachtung über sie ein Widerspruch.
\end{Beweis}


\begin{Theorem}
\label{element_enfuegen}
  Wenn ein Mann $m$ in einer Instanz eine Frau $w$ an seine Prä\-fe\-renz\-lis\-te anhängt, dann gilt weder in der man-optimal noch in der woman-optimal stabilen Paarung für die neue Instanz, dass irgendeine Frau die entsprechende Paarung für die ursprüngliche Instanz bevorzugt und irgendein Mann, bis auf $m$, die entsprechende Paarung für die neue Instanz bevorzugt.
\end{Theorem}

\begin{Beweis}
\label{element_enfuegen_bew}
  Wenn $m$ bereits in der man-optimal Paarung für die ursprüngliche Instanz eine Partnerin hatte, ändert sich nichts, weil $m$ nicht die Gelegenheit bekommt ihr einen Heiratsantrag zu machen. Falls $m$ keine Partnerin hatte, können die Frauen nur profitieren, da sie nur Änderungen akzeptieren die zu einer Verbesserung führen, und die Männer können nur verlieren, indem eine Frau eine Verlobung wieder löst. $m$ bekommt aber die Chance um die Hand von $w$ anzuhalten.\\
  Wenn $w$ bereits in der woman-optimal Paarung für die ursprüngliche Instanz einen Partner hatte, ändert sich nichts, weil $m$ nicht die Gelegenheit bekommt ihr einen Heiratsantrag zu machen. Hatte $w$ keinen Partner, ergibt sich die Gelegenheit $m$ den Hof zu machen. Dies wird zwar angenommen, wenn $m$ noch keine Partnerin hat, kann aber keine andere Frau benachteiligen, weil jede potentielle Geschädigte weiter vorne in der Prioritätenliste von $m$ steht, und $w$ somit immer ersetzen würde.
\end{Beweis}

\subsection{Streng schwach geordnete Präferenzlisten}

Auch diesem Fall, also dem Zulassen von Indifferenz zwischen zwei Männern für eine Frau und umgekehrt, ist es interessant neben der ursprünglichen Definition für Stabilität zwei stärkere zu betrachten.

\begin{Definition}
\label{super-stabil}
  Eine Paarung $P$ ist genau dann nicht \textit{super-stabil}, wenn es zwei Knoten $m \in M$ und $w \in W$ gibt für die gilt, dass $m \leq_{w} p_{P}(w), m \neq p_{P}(w)$ und $w \leq_{m} p_{P}(m), w \neq p_{P}(m)$.
\end{Definition}

\begin{Proposition}
\label{keine_super-stabil}
  Für eine beliebige Instanz des SMP mit streng schwach geordnete Präferenzlisten, kann auch keine super-stabile Paarung existieren.
\end{Proposition}

\begin{Beweis}
\label{keine_super-stabil_bew}
  Für eine Instanz in der $\forall x y. x \leq_{y} x, x \in \textrm{Pr\"aferenzliste}(y)$ gilt, alle Personen auf allen Präferenzlisten also die gleiche Position haben, ist jedes beliebige Paar $(m \in M,w \in W)$ aus jeder möglichen Paarung für diese blockierend.
\end{Beweis}

\begin{Definition}
\label{streng_stabil}
  Eine Paarung $P$ ist genau dann nicht \textit{streng stabil}, wenn es zwei Knoten $m \in M$ und $w \in W$ gibt für die gilt, dass $m \leq_{w} m' = p_{P}(w), m \neq m'$, $w \leq_{m} w' = p_{P}(m), w \neq w'$ und entweder $m <_{w} m'$, oder $w <_{m} w'$ gilt.
\end{Definition}

Der Gale-Shapley Algorithmus jeweils so angepasst werden, damit eine Paarung erzeugt wird, die einer der obigen Definitionen entspricht. Der eigentliche Ablauf ist in beiden Fällen ist gleich, nur werden verschiedene Taktiken angewendet, wenn keine Männer mehr frei sind, es aber noch Personen gibt, die mehrere Verlobte haben.

\begin{algorithm}[H]
  \SetKwData{KwM}{m}\SetKwData{KwM'}{m'}\SetKwData{KwW}{w}
  setze alle Personen auf $frei$\;
  \Solange{$\forall m \in M.\exists w \in W.$ $m$ darf $w$ einen Heiratsantrag machen}{
    \Wenn{ein Mann \KwM keine Partnerin hat}{
      \FuerAlle{Frauen \KwW, die für \KwM höchste Priorität haben und um deren Hand \KwM nicht anhalten hat}{
        \eWenn{\KwW keinen Partner hat}{
          füge \KwW und \KwM zu den jeweiligen Partnerlisten hinzu\;
        }
        {
          \Wenn{\KwW \KwM ihren Verlobten vorzieht, oder indifferent ist}{
            \Wenn{\KwW \KwM ihren Verlobten vorzieht}{
              entferne \KwW aus allen Partnerlisten\;
              leere die Partnerliste von \KwW\;
            }
            füge \KwW und \KwM zu den jeweiligen Partnerlisten hinzu\;
          }
          \lSonst{lehnt \KwW den Antrag ab}
        }
      }
    }
    \lSonst{wende Taktik an}
  }
  \caption{Angepasster Gale–Shapley Algorithmus}
\end{algorithm}


Für super-Stabilität besagt die Taktik, dass jene Frauen, die mehrere Verlobte haben, ihre Verlobungen lösen müssen und auch die anderen Männer gleicher Priorität der jeweiligen Frau keinen Antrag mehr machen dürfen. Wenn der modifizierte Algorithmus nur echte Paare zurück gibt, ist die Paarung super-stabil.\par
Soll die Paarung streng stabil sein, fordert die Taktik, dass die Personen in einem ungerichteter bipartiten Graph die Knoten darstellen und die Verlobungsbeziehungen die Kanten. Lässt sich in diesem Graph ein perfektes Matching bestimmen, so ist dieses auch eine streng stabile Paarung und der Algorithmus terminiert. Im anderen Fall, $\exists \mathcal{M} \subseteq M. |\mathcal{M}| < |\mathcal{W} = \{w | \exists m, w.(m \in \mathcal{M} \land w \in W \land  w \textrm{ ist mit } m \textrm{ verlobt})\}|$, $\mathcal{M}$ wird so gewählt, dass es minimal ist, also keine Menge enthält, die die gleiche Bedingung erfüllt. Für jede Frau in $\mathcal{W}$ mit mehr als einem Verlobten, müssen die Partnerlisten geleert werden und alle gleichpriorisierten Männer dürfen der Frau keinen Heiratsantrag mehr machen.
Wenn der modifizierte Algorithmus nur echte Paare zurück gibt, ist die Paarung streng stabil.\par
Wenn die ursprüngliche Definition \ref{stabil} verlangt wird, können aus den streng schwach geordnete Präferenzlisten, durch beliebige Ordnung der gleichpriorisierten Personen, total geordnete erstellt werden. Die Standardversion vom Algorithmus kann hier angewendet werden und die resultierende Paarung ist stabil. Sie ist aber nicht zwingend man-optimal bezüglich der Ursprungsinstanz.
