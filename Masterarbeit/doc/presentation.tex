\documentclass{beamer}
    \usepackage[utf8]{inputenc}
    \usepackage[T1]{fontenc}
    \usepackage[greek,english]{babel}
    \usepackage{bookmark}
    \usepackage{mathtools}
    \usepackage[compatibility,nofetsolderdot,nooldvoltagedirection,european,betterproportions]{circuitikz}
    \usepackage{tikz}
    \usepackage{pgf}
    \usepackage{pgfplots}
    \usepackage{listings}
    \usepackage{ifthen}
    \usepackage{algorithm}% http://ctan.org/pkg/algorithms
    \usepackage{algpseudocode}% http://ctan.org/pkg/algorithmicx
    \usepackage[backend=biber,isbn=false,doi=false,url=false]{biblatex}

    \usetikzlibrary{positioning, arrows, patterns}

    \usetheme{Berlin}
    \setbeamertemplate{footline}{}
    \setbeamertemplate{headline}{}
    \setbeamertemplate{theorems}[ams style]
    \setbeamertemplate{caption}[numbered]
    % \setbeamercolor{normal text}{bg=gray!5}
    \usecolortheme{lily}
    \usefonttheme{professionalfonts}
    \beamertemplatenavigationsymbolsempty

    % \bibliographystyle{abbrv}
    % \bibliography{cites}

    \title{Formalization of the Isabelle/Pure Logic}
    \author[F. Madge]{Fabio Madge}
    % \institute{Technische Universität München}
    \date[24.02.21]{24 February 2021}

\begin{document}

    \maketitle

    \begin{frame}
    \setcounter{tocdepth}{1}
    \tableofcontents[pausesections]
    \end{frame}

    \AtBeginSection[]
    {
    \begin{frame}
    \setcounter{tocdepth}{2}
    \tableofcontents[currentsection,hideothersubsections]
    \end{frame}
    }

    \AtBeginSubsection[]
    {
    \begin{frame}
    \setcounter{tocdepth}{2}
    \tableofcontents[currentsection,
            currentsubsection,
            subsectionstyle=show/shaded/hide]
    \end{frame}
    }

    \section{Introduction}
    \begin{frame}{Isabelle}
  \begin{itemize}[<+->]
  	\item Generic proof assistant (meta-logic $\mathcal{M}$ + object logic $\mathcal{O}$)
    \item Isabelle/Pure ($\mathcal{M}$): $\bigwedge$, $\Longrightarrow$, and $\equiv$
    \item Isabelle/HOL ($\mathcal{O}$): $\forall$, $\longrightarrow$, and $=$

    \item \texttt{thm}: Abstract datatype encoding a theorem
    \item New theorems from existing ones using natural deduction style inference rules
    \item Kernel: Functionalities modifying the internals of a theorem
    \item LCF approach minimizes size of the critical part of the implementation
    \item Goal: Formalization of Pure in HOL
  \end{itemize}
\end{frame}


    \section{Preliminaries}
    \begin{frame}{Names}
    \begin{isabelle}
        \isacommand{type\_synonym}\ name\ {\isacharequal}\ string \isanewline
        \isacommand{type\_synonym}\ indexname\ {\isacharequal}\ name {\isasymtimes}\ int
    \end{isabelle}
    \begin{itemize}
        \item Logically equivalent
        \item \isa{indexname}: Cheaper generation of fresh names
    \end{itemize}
\end{frame}


\subsection{Types}
\begin{frame}{Polymorphism}
  \begin{itemize}[<+->]
    \item ad-hoc polymorphism (constant overloading)
        \begin{quote}
            \isa{(+) :: 'a {\isasymRightarrow} 'a {\isasymRightarrow} 'a}
        \end{quote}
    \item parametric polymorphism (quantified type variables)
        \begin{quote}
            \Snippet{list}
        \end{quote}
        \phantom{1em}
        \begin{quote}
            \Snippet{list_append}
        \end{quote}
    \item type classes (restricting type variables with sort annotations)
        \begin{quote}
            \begin{isabelle}
                \isacommand{type\_synonym}\ class\ {\isacharequal}\ name \isanewline
                \isacommand{type\_synonym}\ sort\ {\isacharequal}\ class\ set
            \end{isabelle}
        \end{quote}
  \end{itemize}
\end{frame}

\begin{frame}{Unification}
\begin{center}
\scalebox{.94}{
    \prftree[r]{\rs}
    {\left\llbracket A_x\,\middle|\, x \in \left[ m \right] \right\rrbracket \Longrightarrow B}
    {\left\llbracket B_x\,\middle|\, x \in \left[ n \right] \right\rrbracket \Longrightarrow C}
    {B\sigma \equiv B_i\sigma}
    {\left( \left\llbracket B_x\,\middle|\, x \in \left[ i - 1 \right] \right\rrbracket
    + \left\llbracket A_x\,\middle|\, x \in \left[ m \right] \right\rrbracket
    + \left\llbracket B_x\,\middle|\, x \in \left( \left[ n \right] \setminus \left[ i \right] \right) \right\rrbracket
    \Longrightarrow C \right) \sigma
    }
}
\end{center}
\begin{itemize}
    \item Unification integral part of Isabelle
    \item Might instantiate undesired variables
    \item Preserve with special variable kind
\end{itemize}
\end{frame}

\begin{frame}
    \begin{isabelle}
        \isacommand{datatype}\ typ\ {\isacharequal}\ TFree\ name sort\isanewline
        \isaindent{\ \ }{\isacharbar}\ TVar\ indexname\ sort\isanewline
        \isaindent{\ \ }{\isacharbar}\ Type\ name\ {\isacharparenleft}typ\ list{\isacharparenright}
    \end{isabelle}
\end{frame}

\subsection{Term}

% \begin{frame}{De Bruijn Indices}
% \begin{itemize}
%     \item Alternative way of encoding anonymous function using \(\lambda\)-abstractions, e.g. \(\lambda x.\ x + 1\).
%     \item Usually: Binder and bound variable are connected by common name
%     \item De Bruijn: Bound variable holds the distance to its binder
%     \item Good: \(\equiv_\alpha\) becomes structural equality; Potentially smaller memory footprint; No Capturing (\(\lambda y. (\lambda x y.\ f\ x\ y)\ y\))
%     \item Bad: Hard to implement
% \end{itemize}
% \end{frame}

\begin{frame}
    \begin{isabelle}
        \isacommand{datatype}\ term\ {\isacharequal}\ Const name typ\isanewline
        \isaindent{\ \ }{\isacharbar}\ Free name typ\isanewline
        \isaindent{\ \ }{\isacharbar}\ Var indexname typ\isanewline
        \isaindent{\ \ }{\isacharbar}\ Bound nat\isanewline
        \isaindent{\ \ }{\isacharbar}\ Abs typ term\isanewline
        \isaindent{\ \ }{\isacharbar}\ term \$ term
    \end{isabelle}
\end{frame}


\subsection{Validity}

\begin{frame}{Type Signature}
    \begin{quote}
        \begin{isabelle}
            \isacommand{datatype}\ signature\ {\isacharequal}\ Signature\ \isanewline
            \isaindent{\ \ }(const\_typ\_of{\isacharcolon}\ name \isasymrightharpoonup\ typ)\isanewline
            \isaindent{\ \ }(typ\_arity\_of{\isacharcolon}\ name \isasymrightharpoonup\ nat)\isanewline
            \isaindent{\ \ }(sorts\_of{\isacharcolon}\ algebra)
        \end{isabelle}
    \end{quote}
\end{frame}

\begin{frame}
    \begin{quote}
        \begin{isabelle}
            \isacommand{datatype}\ algebra\ {\isacharequal}\ Algebra\ \isanewline
            \isaindent{\ \ }(classes{\isacharcolon}\ class\ rel)\isanewline
            \isaindent{\ \ }(arities{\isacharcolon}\ name \isasymrightharpoonup\ class \isasymrightharpoonup\ sort list)
        \end{isabelle}
    \end{quote}
    \begin{itemize}
        \item Operations on classes/sorts: \isa{class\_leq}, \isa{sort\_ex}, and \isa{sort{\isacharunderscore}leq\ s\isactrlsub {\isadigit{1}}\ s\isactrlsub {\isadigit{2}}\ {\isasymequiv}\ {\isasymforall}c\isactrlsub {\isadigit{2}}{\isasymin}s\isactrlsub {\isadigit{2}}{\isachardot}\ {\isasymexists}c\isactrlsub {\isadigit{1}}{\isasymin}s\isactrlsub {\isadigit{1}}{\isachardot}\ class{\isacharunderscore}leq\ c\isactrlsub {\isadigit{1}}\ c\isactrlsub {\isadigit{2}}}
        \item \isa{algebra\_ok}
            \begin{itemize}
                \item Coregularity: \(c_1 \leq c_2 \Longrightarrow \texttt{arities}\ n\ c_1 \leq \texttt{arities}\ n\ c_2\)
                \item All sorts in the arities exist
            \end{itemize}
    \end{itemize}
\end{frame}


\begin{frame}{Does a given type belong to a sort?}
    \begin{itemize}
        \item \isa{of{\isacharunderscore}sort} \isa{{\isacharcolon}{\isacharcolon}\ algebra\ {\isasymRightarrow}\ typ\ {\isasymRightarrow}\ sort\ {\isasymRightarrow}\ bool}
        \item Trivial for variables
        \item Constructors need to regard the sort obligations
        \item Sort obligations are computed by intersecting the class arities
    \end{itemize}
    % \Snippet{of_sort}
\end{frame}

\begin{frame}{Does a given type adhere to a signature?}
    \begin{itemize}
        \item \isa{sort\_ok'} is just \isa{sort\_ex}
        \item Constants must have the correct number of arguments
    \end{itemize}
    \Snippet{typ_ok'}
\end{frame}

\begin{frame}{Are two types related?}
    \begin{itemize}
        \item Type instantiations must respect the sort obligations
        \item One type is an instance of another, iff it can be matched.
    \end{itemize}
    \Snippet{ty_is_instance}
\end{frame}

\begin{frame}{Does a given term adhere to a signature?}
    \begin{itemize}
        \item \isa{term{\isacharunderscore}ok{\isacharprime}} \isa{{\isacharcolon}{\isacharcolon}} \isa{signature\ {\isasymRightarrow}\ term\ {\isasymRightarrow}\ bool}
        \item Term constant types need to match their most general type
        \item Does no type checking: \isa{typ{\isacharunderscore}of} \isa{{{\isacharcolon}{\isacharcolon}}\ term\ {\isasymRightarrow}\ typ\ option}
    \end{itemize}
    % \Snippet{term_ok'}
\end{frame}

\begin{frame}{Is a signature valid?}
    \begin{quote}
        \begin{isabelle}
            \isacommand{datatype}\ signature\ {\isacharequal}\ Signature\ \isanewline
            \isaindent{\ \ }(const\_typ\_of{\isacharcolon}\ name \isasymrightharpoonup\ typ)\isanewline
            \isaindent{\ \ }(typ\_arity\_of{\isacharcolon}\ name \isasymrightharpoonup\ nat)\isanewline
            \isaindent{\ \ }(sorts\_of{\isacharcolon}\ algebra)
        \end{isabelle}
    \end{quote}
    \begin{itemize}
        \item \isa{algebra\_ok}
        \item Term constant types are valid
        \item Function type has an arity of two
        \item The type constructor arities are in sync with the class arities
    \end{itemize}



    % \begin{quote}
        % \begin{isabelle}%
        % \isacommand{definition} signature{\isacharunderscore}ok\ {\isacharparenleft}Signature\ const{\isacharunderscore}typ\ typ{\isacharunderscore}arity\ sorts{\isacharparenright}\ {\isacharequal}\isanewline
        % \ {\isacharparenleft}{\isasymforall}ty{\isasymin}ran\ const{\isacharunderscore}typ{\isachardot}\ {\isacharparenleft}typ{\isacharunderscore}ok{\isacharprime}\ {\isacharparenleft}Signature\ const{\isacharunderscore}typ\ typ{\isacharunderscore}arity\ sorts{\isacharparenright}{\isacharparenright}\ {\isasymand}\isanewline
        % \isaindent{\ }typ{\isacharunderscore}arity\ {\isacharprime}{\isacharprime}fun{\isacharprime}{\isacharprime}\ {\isacharequal}\ Some\ {\isadigit{2}}\ {\isasymand}\isanewline
        % \isaindent{\ }algebra{\isacharunderscore}ok\ sorts\ {\isasymand}\isanewline
        % \isaindent{\ }dom\ typ{\isacharunderscore}arity\ {\isacharequal}\ dom\ {\isacharparenleft}arities\ sorts{\isacharparenright}\ {\isasymand}\isanewline
        % \isaindent{\ }{\isacharparenleft}{\isasymforall}n{\isasymin}dom\ typ{\isacharunderscore}arity{\isachardot}\isanewline
        % \isaindent{\ {\isacharparenleft}\ \ \ }{\isasymforall}os{\isasymin}ran\ {\isacharparenleft}the\ {\isacharparenleft}arities\ sorts\ n{\isacharparenright}{\isacharparenright}{\isachardot}\ the\ {\isacharparenleft}typ{\isacharunderscore}arity\ n{\isacharparenright}\ {\isacharequal}\ {\isacharbar}os{\isacharbar}{\isacharparenright}{\isacharparenright}%
        % \end{isabelle}
    % \end{quote}
\end{frame}


\subsection{Term Operations}
\begin{frame}
    \begin{itemize}[<+->]
      \item Type instantiation
        \begin{quote}
            \begin{isabelle}
                \isacommand{definition} \isa{instT\ {\isasymsigma}\ t\ {\isasymequiv}\ map{\isacharunderscore}types\ {\isacharparenleft}ty{\isacharunderscore}instT\ {\isasymsigma}{\isacharparenright}\ t}
            \end{isabelle}
        \end{quote}
      \item \isa{abstract\_over v t} replaces all occurrences of \isa{v} in \isa{t} by bound variables
        %   \begin{quote}
            %   \Snippet{abstract_over}
        %   \end{quote}
      \item Function application with potential \(\beta\)-contraction
          \begin{quote}
              \Snippet{betapply}
          \end{quote}
    \end{itemize}
  \end{frame}

    \section{Inductive Predicate}
    \subsection{Theory}

% \begin{frame}{Additional constants}
%     \begin{itemize}
%         \item
%         \begin{quote}
%             \isa{minimal{\isacharunderscore}axioms\ {\isacharequal}\ {\isacharbraceleft}reflexivity{\isacharcomma}\ substitution{\isacharcomma}\ extensionality{\isacharcomma}\ eqI{\isacharbraceright}}
%         \end{quote}

%     \end{itemize}
% % \begin{quote}
%     \begin{isabelle}%
%         \footnotesize
%         \isacommand{definition}\ is{\isacharunderscore}std{\isacharunderscore}sig\ {\isacharparenleft}Signature\ const{\isacharunderscore}typ\ typ{\isacharunderscore}arity\ sorts{\isacharparenright}\ {\isacharequal}\isanewline
%         \isaindent{\ }{\isacharparenleft}typ{\isacharunderscore}arity\ {\isacharprime}{\isacharprime}prop{\isacharprime}{\isacharprime}\ {\isacharequal}\ Some\ {\isadigit{0}}\isanewline
%         \isaindent{\ {\isacharparenleft}}const{\isacharunderscore}typ\ {\isacharprime}{\isacharprime}Pure{\isachardot}eq{\isacharprime}{\isacharprime}\ {\isacharequal}\ Some\ {\isacharparenleft}{\isasymalpha}T\ {\isasymrightarrow}\ {\isasymalpha}T\ {\isasymrightarrow}\ propT{\isacharparenright}\ {\isasymand}\isanewline
%         \isaindent{\ {\isacharparenleft}}const{\isacharunderscore}typ\ {\isacharprime}{\isacharprime}Pure{\isachardot}imp{\isacharprime}{\isacharprime}\ {\isacharequal}\ Some\ {\isacharparenleft}propT\ {\isasymrightarrow}\ propT\ {\isasymrightarrow}\ propT{\isacharparenright}\ {\isasymand}\isanewline
%         \isaindent{\ {\isacharparenleft}}const{\isacharunderscore}typ\ {\isacharprime}{\isacharprime}Pure{\isachardot}all{\isacharprime}{\isacharprime}\ {\isacharequal}\ Some\ {\isacharparenleft}{\isacharparenleft}{\isasymalpha}T\ {\isasymrightarrow}\ propT{\isacharparenright}\ {\isasymrightarrow}\ propT{\isacharparenright}{\isacharparenright}%
%         \end{isabelle}

%     % \end{quote}
% \end{frame}

% \begin{frame}{Axioms}
%     \Snippet{minimal_axioms}
% \end{frame}

\begin{frame}{Is a theory valid?}
    \begin{quote}
        \begin{isabelle}
            \isacommand{datatype}\ theory\ {\isacharequal}\ Theory\isanewline
            \isaindent{\ \ }{\isacharparenleft}signature\_of: signature{\isacharparenright}\isanewline
            \isaindent{\ \ }{\isacharparenleft}axioms\_of: term set{\isacharparenright}
        \end{isabelle}
    \end{quote}
    \begin{itemize}
        \item \isa{signature\_ok}
        \item Axiomatizes equality: refl, subst, ext, and eqI
        \item \isa{prop} type and logical constants
        \item The axioms are valid terms of type \isa{prop}
    \end{itemize}
\end{frame}

\subsection{Theorem}

\begin{frame}
    \begin{itemize}
        \item Valid theory\ \(\Theta\)
        \item Set of hypotheses\ \(\Gamma\) of type \isa{prop}
        \item Hypotheses must be free of schematic variables: \isa{no\_vars A \isasymequiv\ Vars A = \{\} \isasymand\ TVars A = \{\}}
        \item Proposition \isa{A}
        \item Typeset as \(\Theta, \Gamma \vdash\) \isa{A}

    \end{itemize}
\end{frame}

\subsection{Inference Rules}
\begin{frame}
    \begin{isabelle}
    \begin{center}{
        \isa{\mbox{}\inferrule{\mbox{theory{\isacharunderscore}ok\ {\isasymTheta}}\\\ \mbox{instT{\isacharunderscore}ok\ {\isasymTheta}\ {\isasymsigma}}\\\ \mbox{A\ {\isasymin}\ axioms{\isacharunderscore}of\ {\isasymTheta}}}{\mbox{{\isasymTheta}{\isacharcomma}{\isasymemptyset}\ {\isasymturnstile}\ instT\ {\isasymsigma}\ A}}} {\textsc{axiom}}\\[1ex]
        \mprset{sep=1.6em}
        \isa{\mbox{}\inferrule{\mbox{theory{\isacharunderscore}ok\ {\isasymTheta}}\\\ \mbox{term{\isacharunderscore}ok\ {\isasymTheta}\ A}\\\ \mbox{no{\isacharunderscore}vars\ A}\\\ \mbox{typ{\isacharunderscore}of\ A\ {\isacharequal}\ Some\ propT}}{\mbox{{\isasymTheta}{\isacharcomma}{\isacharbraceleft}A{\isacharbraceright}\ {\isasymturnstile}\ A}}} {\textsc{assume}}\\[1ex]
        \mprset{sep=2.0em}
        \isa{\mbox{}\inferrule{\mbox{{\isasymTheta}{\isacharcomma}{\isasymGamma}\ {\isasymturnstile}\ A}\\\ \mbox{typ{\isacharunderscore}ok\ {\isasymTheta}\ {\isasymtau}}}{\mbox{{\isasymTheta}{\isacharcomma}{\isasymGamma}\ {\isasymturnstile}\ mk{\isacharunderscore}all\ {\isacharparenleft}Var\ x\ {\isasymtau}{\isacharparenright}\ A}}} {\textsc{forallI\(_{\mbox{var}}\)}}\\[1ex]
        \mprset{sep=1.2em}
        \isa{\mbox{}\inferrule{\mbox{{\isasymTheta}{\isacharcomma}{\isasymGamma}\ {\isasymturnstile}\ A}\\\ \mbox{{\isacharparenleft}x{\isacharcomma}\ {\isasymtau}{\isacharparenright}\ {\isasymnotin}\ Frees{\isacharunderscore}Set\ {\isasymGamma}}\\\ \mbox{typ{\isacharunderscore}ok\ {\isasymTheta}\ {\isasymtau}}}{\mbox{{\isasymTheta}{\isacharcomma}{\isasymGamma}\ {\isasymturnstile}\ mk{\isacharunderscore}all\ {\isacharparenleft}Free\ x\ {\isasymtau}{\isacharparenright}\ A}}} {\textsc{forallI\(_{\mbox{free}}\)}}\\[1ex]
        \mprset{sep=2.0em}
        \isa{\mbox{}\inferrule{\mbox{{\isasymTheta}{\isacharcomma}{\isasymGamma}\ {\isasymturnstile}\ all{\isacharunderscore}const\ {\isasymtau}\ {\isachardollar}\ A}\\\ \mbox{term{\isacharunderscore}ok\ {\isasymTheta}\ x}\\\ \mbox{typ{\isacharunderscore}of\ x\ {\isacharequal}\ Some\ {\isasymtau}}}{\mbox{{\isasymTheta}{\isacharcomma}{\isasymGamma}\ {\isasymturnstile}\ A\ {\isasymbullet}\ x}}} {\textsc{forallE}}
    }
    \end{center}
\end{isabelle}
\end{frame}

\begin{frame}
    \begin{isabelle}
    \begin{center}{
        \isa{\mbox{}\inferrule{\mbox{{\isasymTheta}{\isacharcomma}{\isasymGamma}\ {\isasymunion}\ {\isacharbraceleft}A{\isacharbraceright}\ {\isasymturnstile}\ B}}{\mbox{{\isasymTheta}{\isacharcomma}{\isasymGamma}\ {\isasymturnstile}\ A\ {\isasymlongmapsto}\ B}}} {\textsc{impliesI}}\\[1ex]
        \isa{\mbox{}\inferrule{\mbox{{\isasymTheta}{\isacharcomma}{\isasymGamma}\isactrlsub {\isadigit{1}}\ {\isasymturnstile}\ A\ {\isasymlongmapsto}\ B}\\\ \mbox{{\isasymTheta}{\isacharcomma}{\isasymGamma}\isactrlsub {\isadigit{2}}\ {\isasymturnstile}\ A}}{\mbox{{\isasymTheta}{\isacharcomma}{\isasymGamma}\isactrlsub {\isadigit{1}}\ {\isasymunion}\ {\isasymGamma}\isactrlsub {\isadigit{2}}\ {\isasymturnstile}\ B}}} {\textsc{impliesE}}\\[1ex]
        \isa{\mbox{}\inferrule{\mbox{theory{\isacharunderscore}ok\ {\isasymTheta}}\\\ \mbox{term{\isacharunderscore}ok\ {\isasymTheta}\ {\isacharparenleft}Abs\ {\isasymtau}\ A\ {\isachardollar}\ x{\isacharparenright}}}{\mbox{{\isasymTheta}{\isacharcomma}{\isasymemptyset}\ {\isasymturnstile}\ mk{\isacharunderscore}eq\ {\isacharparenleft}Abs\ {\isasymtau}\ A\ {\isachardollar}\ x{\isacharparenright}\ {\isacharparenleft}Abs\ {\isasymtau}\ A\ {\isasymbullet}\ x{\isacharparenright}}}} {\textsc{\(\beta\)\_conv}}\\[1ex]
        \mprset{sep=1.6em}
        \isa{\mbox{}\inferrule{\mbox{{\isasymTheta}{\isacharcomma}{\isasymGamma}\ {\isasymturnstile}\ B}\\\ \mbox{term{\isacharunderscore}ok\ {\isasymTheta}\ A}\\\ \mbox{no{\isacharunderscore}vars\ A}\\\ \mbox{typ{\isacharunderscore}of\ A\ {\isacharequal}\ Some\ propT}}{\mbox{{\isasymTheta}{\isacharcomma}{\isasymGamma}\ {\isasymunion}\ {\isacharbraceleft}A{\isacharbraceright}\ {\isasymturnstile}\ B}}} {\textsc{weaken}}
        \mprset{sep=2.0em}}
      \end{center}
    \end{isabelle}
\end{frame}

\begin{frame}{Properties}
    \begin{theorem}
        \InlineLemma{proves_theory_ok}
    \end{theorem}
    \begin{theorem}
        \InlineSnippet{thm_is_prop}
    \end{theorem}
    \begin{proof}
        Rule induction on \isa{\(\Gamma\)} and \isa{A}
    \end{proof}
\end{frame}

    \section{Derived Rules}
    \subsection{Term Instantation}

\begin{frame}
    \begin{lemma}
        \InlineLemma{instantiate_var_same_typ'}
    \end{lemma}
    \begin{proof}
        Structural induction on \isa{B} after appropriate generalization.
    \end{proof}

    \begin{theorem}
        \InlineLemma{inst_var}
    \end{theorem}
    \begin{proof}
        \begin{itemize}
            \item \InlineSnippet{inst_var_1} (\aIv)
            \item \InlineSnippet{inst_var_2} (\aE)
        \end{itemize}
    \end{proof}
\end{frame}

\begin{frame}{Parallel Instantation}
    \begin{itemize}
        \item Problem: \(x [a/x, b/a] \neq x [a/x][b/a]\)
        \item Overlap of the left and right sides of the substitution
        \item Solution: rename the offending variables in the substituters
    \end{itemize}
\end{frame}

\subsection{Type Instantation}
\begin{frame}{Weaken \isa{of\_sort}}
    \begin{theorem}
        \InlineLemma{weaken_of_sort}
    \end{theorem}
    \begin{corollary}
        \InlineLemma{of_sort_ty_instT}
    \end{corollary}
    \begin{proof}
        \begin{itemize}
            \item Structural induction on \(\tau\)
            \item Case: \isa{TVar n S'}
            \begin{itemize}
                \item \isa{TVar n S'} belongs to sort \(S\) \(\Longrightarrow S' \leq S\)
                \item \isa{instT\_ok} \(\Longrightarrow \sigma (n, S')\) belongs to sort \(S'\)
                \item Theorem~\ref{lemma:weaken_of_sort}
            \end{itemize}
        \end{itemize}
    \end{proof}
\end{frame}

\begin{frame}{Compose Instantiation (1)}
    \begin{quote}
        \begin{isabelle}
            \isacommand{definition} \isa{f\ {\isasymcirc}\isactrlsub i\isactrlsub n\isactrlsub s\isactrlsub t\isactrlsub T\ g\ {\isacharequal}\ {\isacharparenleft}{\isasymlambda}k{\isachardot}\ \textsf{case}\ g\ k\ \textsf{of}\isanewline
            \isaindent{\ \ \ }None\ {\isasymRightarrow}\ f\ k\isanewline
            \isaindent{\ }{\isacharbar}\ Some\ T\ {\isasymRightarrow}\ Some\ {\isacharparenleft}ty{\isacharunderscore}instT\ f\ T{\isacharparenright}{\isacharparenright}}
        \end{isabelle}
    \end{quote}
    \begin{lemma}
        \InlineLemma{collapse_instT}
    \end{lemma}
    \begin{theorem}
        \InlineLemma{instT_term_ok'}
    \end{theorem}
    \begin{proof}
        \begin{itemize}
            \item Structural induction on \isa{A}
            \item Case: \isa{Const n \isamath{\tau}}
            \begin{itemize}
                \item Map \(\sigma^\prime\) from \(\tau\) to polymorphic type of \isa{n} exits
                \item \isa{{\isasymsigma}\ {\isasymcirc}\isactrlsub i\isactrlsub n\isactrlsub s\isactrlsub t\isactrlsub T\ {\isasymsigma}{\isacharprime}} preserves \isa{instT\_ok}
            \end{itemize}
        \end{itemize}
    \end{proof}
\end{frame}

\begin{frame}{Compose Instantiation (2)}
    \begin{corollary}
        \InlineLemma{instT_term_ok}
    \end{corollary}
\end{frame}

\begin{frame}{Alternative Inference Rules}
    \begin{itemize}
        \item Observation: Not all variables used to derive a theorem are in the proposition
        \item Problem: No way to tell if capturing may occur
        \item Solution: Keep track of all variables used
    \end{itemize}
    \begin{center}{
    \isa{\mbox{}\inferrule{\mbox{{\isasymTheta}{\isacharcomma}{\isasymGamma}{\isacharcomma}{\isasymPsi}{\isacharcomma}{\isasymOmega}\ {\isasymturnstile}\ A}\\\ \mbox{typ{\isacharunderscore}ok\ {\isasymTheta}\ {\isasymtau}}}{\mbox{{\isasymTheta}{\isacharcomma}{\isasymGamma}{\isacharcomma}{\isasymPsi}\ {\isasymunion}\ {\isacharbraceleft}{\isacharparenleft}x{\isacharcomma}\ {\isasymtau}{\isacharparenright}{\isacharbraceright}{\isacharcomma}{\isasymOmega}\ {\isasymturnstile}\ mk{\isacharunderscore}all\ {\isacharparenleft}Var\ x\ {\isasymtau}{\isacharparenright}\ A}}} \aIv}
    \end{center}
    \begin{lemma}
        \InlineLemma{vars_varhyps}
    \end{lemma}
    \begin{lemma}
        \InlineLemma{frees_freehyps}
    \end{lemma}
    \begin{Theorem}
        \InlineLemma{proves_with_variables_eqivalent}
    \end{Theorem}
\end{frame}

\begin{frame}{\isa{instT} Theorem}
    \Snippet{subst_flat}
    \begin{theorem}
        \InlineLemma{instT_proves_with_variables_flat_subst}
    \end{theorem}
\end{frame}

\begin{frame}
    \begin{theorem}
        \InlineLemma{instT_proves_with_variables_flat_subst}
    \end{theorem}
    \begin{corollary}
        \begin{isabelle}
            \isa{{\isasymlbrakk}{\isasymTheta}{\isacharcomma}{\isasymGamma}\ {\isasymturnstile}\ A{\isacharsemicolon}\ instT{\isacharunderscore}ok\ {\isasymTheta}\ {\isasymsigma}{\isasymrbrakk}\ {\isasymLongrightarrow}\ {\isasymTheta}{\isacharcomma}{\isasymGamma}\ {\isasymturnstile}\ instT\ {\isasymsigma}\ A}
        \end{isabelle}
    \end{corollary}
    \begin{proof}
        \begin{enumerate}
            \item Flatten substitution by renaming
            \item Convert to tracked theorem
            \begin{enumerate}
                \item Rename captured variables in the map
                \item Theorem~\ref{lemma:instT_proves_with_variables_flat_subst}
            \end{enumerate}
            \item Reconvert
            \item Revert renamings
        \end{enumerate}
    \end{proof}
\end{frame}

    \section{Conclusion}
    \subsection{Still Missing}
\begin{frame}
  \begin{itemize}[<+->]
  	\item Performance Optimization
    \item Elim-resolution
    \item More elementary renaming
    \item Handling of flex-flex pairs
    \item Support for proof terms
  \end{itemize}
\end{frame}

\subsection{New Opportunities}
\begin{frame}
  \begin{itemize}[<+->]
	\item Change the existing HOU
    \item Add a true FOU (blast)
    \item Correctness proof
    \item ?
  \end{itemize}
\end{frame}


\end{document}
