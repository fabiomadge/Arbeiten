% !TeX root = ../main.tex
% Add the above to each chapter to make compiling the PDF easier in some editors.

\chapter{Introduction}\label{chapter:introduction}

The goal of proof assistants is to enable the use of rigorous formal arguments on a big scale with the help of software.
There are different ways of designing such a system, one of them pioneered by the LCF prover.
The programs that belong to the LCF family~\parencite{Gordon2000}, all define an abstract data type to represent theorems and allows them to be combined using inference rules to yield new theorems.
Only inference rules can create theorems because the type hides their internals outside of the so-called kernel.
It is comprised of the logical primitives and the functionalities needed to implement them, including term manipulation and type checking.
The automation that makes the system usable, in turn, uses the functionalities offered by the kernel.
Following this design ensures that only a small part of the very complex software needs to be trusted because only that part is a possible vector for inconsistencies (\(A \land \neg A\)).\\

Out the LCF family programs, Isabelle is at present one of most frequently used.
Its feature of separating the meta-logic \(\mathcal{M}\)~\parencite{Paulson1988} and an object-logic \(\mathcal{O}\) sets Isabelle apart.
The result is a generic system that allows some functionalities to be reused by embedding the desired object-logic into \(\mathcal{M}\).
Examples of embedded object-logics include first-order logic, Zermelo–Fraenkel set theory, and higher-order logic (HOL).
HOL is the most popular choice amongst users, to the point that they often use the names Isabelle and Isabelle/HOL interchangeably.
In recent years, the implementation of \(\mathcal{M}\) has begun to be referred to as Isabelle/Pure.
It underwent some changes in its infancy but has been stable for many years.
This thesis aims to formalize \(\mathcal{M}\) within Isabelle/HOL, which not only presents an accessible version of the kernel's logical foundations, but also a stepping stone for further developments, like a verified version of Isabelle/Pure.\\

At first, we present some definitions in Chapter~\ref{chapter:preliminaries}, including the term and type languages and operations on these languages.
Using these definitions, we construct the deductive system in Chapter~\ref{chapter:predicate}, and finally, show some derived rules in Chapter~\ref{chapter:derived}.