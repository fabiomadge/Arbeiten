% !TeX root = ../main.tex
% Add the above to each chapter to make compiling the PDF easier in some editors.

\chapter{Introduction}\label{chapter:introduction}

Isabelle is a proof assistant and belongs to the LCF family~\parencite{Gordon2000}.
Their goal is to enable to use of rigorous formal arguments on a big scale.
As its peers, it defines an abstract data type representing theorems.
Combining those using inference rules yields new theorems.
Only those can create theorems because the abstract data type hides their internals outside of the so-called kernel.
It comprises the logical primitives and the functionality needed to implement them.
These include things like term manipulation and type checking. The automation making the system usable, in turn, uses the functionality offered by the kernel.
Following this design ensures that only a small part of the very complex software needs to be trusted because only that is a possible vector for inconsistencies.\\

Separating the meta-logic \(\mathcal{M}\)~\parencite{Paulson1988}, and an object-logic \(\mathcal{O}\) sets Isabelle apart.
The result is a generic system that allows reusing some functionality by embedding the desired object-logic into \(\mathcal{M}\).
Examples include first-order logic, Zermelo–Fraenkel set theory, and higher-order logic (HOL).
It is the most popular choice amongst users, to the point that they often use Isabelle and Isabelle/HOL interchangeably.
These days the implementation of \(\mathcal{M}\) has a name and goes by Isabelle/Pure.
It underwent some changes in its infancy but has been stable for many years.
This thesis aims to formalize \(\mathcal{M}\) within Isabelle/HOL, a worthwhile endeavor in its own right.
It presents an accessible version of the kernel's logical foundations to anyone interested but can also be a stepping stone for further developments, like a verified version of Isabelle/Pure.\\

At first, we present some preliminaries, including the term and type languages and operations on those, using those, construct the deductive system, and finally show some derived rules.