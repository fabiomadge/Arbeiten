\chapter{Deductive System}\label{chapter:predicate}

Once we have theorems, we can define our inference rules and ultimately prove some properties about the resulting system.

\section{Theory}

We need to introduce some constants with the goal in mind of using the terms to encode Pure propositions.
First, reserve the type constant ``pro`` for those propositions, which strongly correlates to the more common boolean type, at least in a classical logic.
We also introduce some abbreviations, namely \isa{propT \isasymequiv\ Type {\isacharprime}{\isacharprime}prop{\isacharprime}{\isacharprime} []} and \isa{\(\alpha\)T = TVar ({\isacharprime}{\isacharprime}{\isacharprime}a{\isacharprime}{\isacharprime}, 0) \{\}}.
To simplify dealing with term variables, we introduce \isa{VAR x T \isasymequiv\ Var (x, 0) T}.
For equality we  use \isa{mk\_eq' \(\tau\) t1 t2 \isasymequiv\ Const {\isacharprime}{\isacharprime}Pure.eq{\isacharprime}{\isacharprime} (\(\tau\) {\isasymrightarrow} \(\tau\) {\isasymrightarrow} propT) \$ t1 \$ t2}, and \isa{mk\_eq}, which uses typ\_of to determine the argument type.
To use the implications we use the infix abbreviation \isa{A {\isasymlongmapsto} B \isasymequiv\ Const {\isacharprime}{\isacharprime}Pure.imp{\isacharprime}{\isacharprime} (propT {\isasymrightarrow} (propT {\isasymrightarrow} propT)) \$ A \$ B}
The quantifier's argument is a function that expects a value for the quantified variable.
Using \isa{all\_const T \isasymequiv\ Const {\isacharprime}{\isacharprime}Pure.all{\isacharprime}{\isacharprime} ((T {\isasymrightarrow} propT) {\isasymrightarrow} propT)}, we can get the constant and \isa{mk\_all x t \isasymequiv\ all\_const (the (typ\_of x)) \$ mk\_Abstraction x t} lets us abbreviate the whole process of universal quantification.
We use \isa{is\_std\_sig} to verify that a given signature defines all those constants as indicated.

% \Snippet{is_std_sig}
\begin{quote}
\begin{isabelle}%
    \isacommand{definition}\ is{\isacharunderscore}std{\isacharunderscore}sig\ {\isacharparenleft}Signature\ const{\isacharunderscore}typ\ typ{\isacharunderscore}arity\ sorts{\isacharparenright}\ {\isacharequal}\isanewline
    \isaindent{\ }{\isacharparenleft}typ{\isacharunderscore}arity\ {\isacharprime}{\isacharprime}prop{\isacharprime}{\isacharprime}\ {\isacharequal}\ Some\ {\isadigit{0}}\ {\isasymand}\ typ{\isacharunderscore}arity\ {\isacharprime}{\isacharprime}fun{\isacharprime}{\isacharprime}\ {\isacharequal}\ Some\ {\isadigit{2}}\ {\isasymand}\isanewline
    \isaindent{\ {\isacharparenleft}}const{\isacharunderscore}typ\ {\isacharprime}{\isacharprime}Pure{\isachardot}eq{\isacharprime}{\isacharprime}\ {\isacharequal}\ Some\ {\isacharparenleft}{\isasymalpha}T\ {\isasymrightarrow}\ {\isasymalpha}T\ {\isasymrightarrow}\ propT{\isacharparenright}\ {\isasymand}\isanewline
    \isaindent{\ {\isacharparenleft}}const{\isacharunderscore}typ\ {\isacharprime}{\isacharprime}Pure{\isachardot}imp{\isacharprime}{\isacharprime}\ {\isacharequal}\ Some\ {\isacharparenleft}propT\ {\isasymrightarrow}\ propT\ {\isasymrightarrow}\ propT{\isacharparenright}\ {\isasymand}\isanewline
    \isaindent{\ {\isacharparenleft}}const{\isacharunderscore}typ\ {\isacharprime}{\isacharprime}Pure{\isachardot}all{\isacharprime}{\isacharprime}\ {\isacharequal}\ Some\ {\isacharparenleft}{\isacharparenleft}{\isasymalpha}T\ {\isasymrightarrow}\ propT{\isacharparenright}\ {\isasymrightarrow}\ propT{\isacharparenright}{\isacharparenright}%
    \end{isabelle}
\end{quote}

Up to this point, we have only reserved the constants' names but have yet to give them their intuitive meaning within the logic by axiomatizing them.
Because implication and universal quantification have entanglements beyond the proposition, we deal with them using inference rules.
For equality, we use the axiomatization sketched in~\parencite{implementation}, which gives us a minimal set of axioms for the logic.

\Snippet{minimal_axioms}

The theory combines the type signature with the axioms of the system.
We bring all operations on signatures to theorems naturally and mark them by appending \isa{\_thy}.
The \isa{term\_ok\_thm} predicate adds a welltypedness requirement.

\begin{quote}
    \begin{isabelle}
        \isacommand{datatype}\ theory\ {\isacharequal}\ Theory\isanewline
        \isaindent{\ \ }{\isacharparenleft}signature\_of: signature{\isacharparenright}\isanewline
        \isaindent{\ \ }{\isacharparenleft}axioms\_of: term set{\isacharparenright}
    \end{isabelle}
\end{quote}

We only have a limited notion of a valid theory.
While we require a valid signature that correctly defines our constants, the axioms are more challenging to get right.
We make sure the axioms contain the equality axioms, are valid, and have the proposition type, but have no way to verify their consistency.

% \Snippet{theory_ok}
\begin{quote}
\begin{isabelle}%
    \isacommand{definition}\ theory{\isacharunderscore}ok\ {\isacharparenleft}Theory\ {\isasymSigma}\ axioms{\isacharparenright}\ {\isacharequal}\isanewline
    \isaindent{\ }signature{\isacharunderscore}ok\ {\isasymSigma}\ {\isasymand}\ is{\isacharunderscore}std{\isacharunderscore}sig\ {\isasymSigma}\ {\isasymand}\isanewline
    \isaindent{\ }minimal{\isacharunderscore}axioms\ {\isasymsubseteq}\ axioms\ {\isasymand}\isanewline
    \isaindent{\ }{\isacharparenleft}{\isasymforall}p{\isasymin}axioms{\isachardot}\ term{\isacharunderscore}ok{\isacharprime}\ {\isasymSigma}\ p\ {\isasymand}\ typ{\isacharunderscore}of\ p\ {\isacharequal}\ Some\ propT{\isacharparenright}{\isacharparenright}%
\end{isabelle}
\end{quote}

\section{Theorem}

Theorems are judgments in our natural deduction style system.
They combine a theory\ \(\Theta\) and a set of hypotheses\ \(\Gamma\) with a proposition \isa{A}, and are typeset \(\Theta, \Gamma\ \isasymturnstile\ \isa{A}\).
We expect any theorem to have a valid theory.
The hypotheses must all be of type \isa{prop}.
To ensure that instantiations do not affect the hypotheses, we ban schematic variables from them.
We define functions that extract the instances of specific constructors in a given term or type.
They share a name but capitalize the first letter.
Those operating on types have a capital \isa{T} appended.
This way we can express the absence of schematic variables as \isa{no\_vars A \isasymequiv\ Vars A = \{\} \isasymand\ TVars A = \{\}}.
The inference rules described next guarantee those properties by construction, as we will prove later.

\section{Inference Rules}

A proof tree can represent any proof in our system.
All nodes are instances of our inference rules, but the leaves will either use \ax\ or \as.
As we will later see, term instantiation is a consequence of our system, but the same is not right for types.
There are different ways of gaining that capability, but we choose to strengthen \ax.
While this rule is typically only used to retrieve axioms from the theory to make them a theorem, we allow arbitrary type instantiations.
\as\ lets us assume any proposition, with the caveat of adding it to the hypothesis.
As a result, it has to be free of schematic variables.
Instead of further complicating \ax\ and \as\ by enabling the superfluous addition hypothesis, we allow weakening at arbitrary stages of a proof using \wk.
All variables are implicitly universally quantified, but we can also make that explicit with the universal quantifier.
There are two types of variables that require separate handling, so we have two rules.
We only attach the quantifier to the proposition, so we need to consider the potential for renaming a variable that is both in the proposition and the hypotheses apart.
Because schematic variables cannot occur in the hypotheses, we need no restrictions for \aIv.
Adding a \isa{\_Set} to the extraction functions naturally extends them to sets.
With the help of \isa{Frees\_Set} we can state \aIf.
To eliminate the quantifier, we can use \aE, which instantiates the variable with some term with the matching type.
Using \iI, we can remove a hypothesis by attaching it to the proposition in an implication.
To remove an implication using \iE, we need to prove its premise.
\bcnv\ is an instance of \ax.
It is needed to generate theorems for that axiom because general \(\beta\)-contraction can't be expressed in a proposition.

\Snippet{proves}

\section{Properties}
Before diving right into things, we will briefly discuss how to prove theorems involving judgments.
Knowing that judgments are always a product of the inference rules, we get an induction rule with a case for every inference rule.
In every case, we replace the judgment with the premises needed to apply the rule and instantiate \(\Gamma\) and \isa{A} with values they have in the respective rules.
For every judgment in the premise of a rule, we get an induction hypothesis with appropriate instantiations.
Even though it is somewhat unwieldy, we show the induction rule in all its glory.

\Snippet{induct}

To show some property involving \(\Gamma\) and \isa{A}, we need there to be a judgment proving one from the other in some theory.
We must then show that P holds for every case.
As promised, we first show that any theorem has a valid theory and the proposition is of type \isa{prop}.

\begin{theorem}
    \InlineLemma{proves_theory_ok}
\end{theorem}
\begin{proof}
    Proof by rule induction. All cases can be solved by assumption.
    \ax, \as, and \bcnv\ have \isa{theory\_ok \(\Theta\)} in the hypotheses of their rule, while the others get it from their respective induction hypotheses.
\end{proof}

Proving welltypedness of propositions is, unfortunately, significantly more involved.
It requires some lemmata about the interaction of \isa{typ\_of} and the various term operations.

\begin{lemma}
\label{lemma:redundant}
    Given \isa{typ\_of1 bs (f \$ u) = Some \(\beta\)}, \isa{typ\_of1 bs u = Some \(\alpha\)}, and \isa{typ\_of1 bs f = Some (\(\alpha\ \isasymrightarrow\ \beta\))}, each follows from the remaining ones.
\end{lemma}
\begin{proof}
    Follows from the definition of \isa{typ\_of1}.
\end{proof}

\begin{lemma}
    \InlineLemma{instT_preserve_typ}
\end{lemma}
\begin{proof}
Proof by structural induction on \isa{A}.
Apart from the application \isa{f \$ x}, all cases follow immediately from the definition.
Here, we get \isa{f \$ x} \(: \alpha\), which we need to split in order to use the induction hypotheses.
Using that, we can obtain a type \(\beta\), such that \isa{f} \(: \beta\ \isasymrightarrow\ \alpha\), which gives us \isa{x} \(: \beta\).
\end{proof}

\begin{lemma}
    \InlineLemma{abstract_over1_typ}
\end{lemma}
\begin{proof}
    Proof by structural induction on \isa{A}.
    \begin{description}[]
        \item [\isa{Abs \(\beta\) A}] We first split \(\alpha\) again to obtain \(\beta\ \isasymrightarrow\ \gamma\). We give the call to \isa{abstract\_over1} the name \isa{A'} and then show \InlineSnippet{abstract_over1_typ_1}. The induction hypothesis gives us \InlineSnippet{abstract_over1_typ_2}. Now we just reduce the goal to use these facts.
        \item [\isa{f \$ u}] We again split the type. Knowing that the application is unequal to the variable, we can use the IH's to solve the goal.
    \end{description}
\end{proof}

Generalizing \isa{subst\_bound1} and \isa{incr\_bv} is remarkable because it requires combining a global context \isa{gs} with \isa{ls}, the local accumulator of \isa{subst\_bound1}.
The order of the binders in the accumulators is inside out.

\begin{lemma}
    \InlineLemma{typ_of1_incr_bv}
\end{lemma}
\begin{proof}
    Proof by structural induction on t. Appart from some list arithmetic, the cases all follow from the definitions.
\end{proof}


\begin{lemma}
    \InlineLemma{typ_of1_subst_bound}
\end{lemma}
\begin{proof}
    Proof by induction on the call to \isa{subst\_bound1}. In the \isa{Bound} case we use Lemma~\ref{lemma:typ_of1_incr_bv}.
\end{proof}


\begin{theorem}
    \InlineSnippet{thm_is_prop}
\end{theorem}
\begin{proof}
    Proof by rule induction on \isa{\(\Gamma\)} and \isa{A}.
    \begin{description}[]
        \item [\ax\ \isa{\(\sigma\) A}] We know \isa{A}'s type, because it's an axiom. Using Lemma~\ref{lemma:instT_preserve_typ} we can then solve the case.
        \item [\aIv\ \& \aIf] Lemma~\ref{lemma:abstract_over1_typ} is sufficient.
        \item [\aE] From the IH we get \isa{A} \(: \tau\ \isasymrightarrow\ \mathit{prop}\). Using that, we can apply Lemma~\ref{lemma:typ_of1_subst_bound}.
        \item [\iI] We have \isa{B} \(: \mathit{prop}\), because \isa{B} is in the hypotheses. All that is needed now to prove the goal is the IH.
        \item [\iE] Because \isa{A} \(: \mathit{prop}\) (IH) holds, we also get \isa{(\isasymlongmapsto) A} \(: \mathit{prop}\ \isasymrightarrow\ \mathit{prop}\) for the partially applied implication. Using the other IH and Lemma~\ref{lemma:redundant} we prove the goal.
        \item [\bcnv\ \isa{\(\alpha\) A x}] We initially observe that \isa{x} \(: \alpha\) holds. Then we obtain \(\beta\) such that \isa{Abs \(\alpha\) A \$ x} \(: \beta\). Combining those using Lemma~\ref{lemma:redundant} gives us \isa{Abs \(\alpha\) A} \(: \alpha\ \isasymrightarrow\ \beta\). Ultimately we have \isa{subst\_bound x A} \(: \beta\) using lemma~\ref{lemma:typ_of1_subst_bound}. These two reductions to \(\beta\) suffice to prove the goal.
    \end{description}
\end{proof}

By now, we know that any proposition inside a theorem has the correct type, but we left out \isa{term\_ok}.
It requires a comparable proof and more material on type instantiation, which we will develop in the next chapter.