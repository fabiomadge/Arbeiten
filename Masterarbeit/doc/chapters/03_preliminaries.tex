\chapter{Preliminaries}\label{chapter:preliminaries}

The general goal of this formalization is to match the current implementation of the logic where possible, while avoiding unnecessary complexity.
First, we will discuss the term language of our logic, which we categorize by employing a type system.
However, before examining types and terms, we need to consider names.

\begin{quote}
    \begin{isabelle}
        \isacommand{type\_synonym}\ name\ {\isacharequal}\ string \isanewline
        \isacommand{type\_synonym}\ indexname\ {\isacharequal}\ name {\isasymtimes}\ int
    \end{isabelle}
\end{quote}

Isabelle extends the canonical way of encoding names.
Instead of just using strings, it uses the concept of index names, which entails attaching an integer to the \isa{name}.
The rationale for using index names in an implementation is purely practical, as both \isa{name} and \isa{int} are infinite on their own, at least in the context of PolyML, the only compiler capable of building the current implementation.
Keeping track of the maximum index used within a particular environment simplifies the problem of obtaining fresh names.
Instead of traversing the whole term to ensure uniqueness, we merely increment the maximum index of the term.
Combining the resulting index with any string is then guaranteed to yield a unique index name.
The cost of including index names in the logic is negligible, but it avoids formalizing a construction to embed index names into regular ones.

\section{Types}
Like all the other LCF family members' underlying logics, Pure is a generalization of simply typed lambda calculus~\parencite{Church40}, so it is only fitting to start by talking about its type language.
Isabelle's take on polymorphism makes it unique.
As previously stated, Isabelle is the only member of the LCF family with support for ad-hoc polymorphism.
Here, the user declares a term constant, the type of which contains variables, and later supplies definitions for different instances of that type.
An example of that is addition using the constant \isa{(+) :: 'a {\isasymRightarrow} 'a {\isasymRightarrow} 'a}.
It is immediately apparent that the addition of integers must behave differently than the addition of reals.
Therefore we need separate definitions for these operations, but reusing the \isa{(+)} constant for them is convenient and thus desirable.

All HOL provers support parametric polymorphism up to Rank-1, which allows for abstracting over unnecessarily specific type information.
\isa{'a list}, an abstract data type that encodes a sequence of arbitrary elements, exemplifies that.
\Snippet{list}
While the element type is irrelevant to the structure, the \isa{'a} can not persist in a concrete list object.
Abstract data types become especially useful when combined with abstract function definitions.
The following abstract definition of concatenation is oblivious to the element type and works for any two lists of the same type.
\Snippet{list_append}

Isabelle takes this a step further by allowing restrictions on type variables~\parencite{Nipkow93}.
So-called sorts, representing subsets of all defined types, are added to the variable.
Instantiating those annotated variables is only possible with types that belong to that particular sort.
To compare sorts, we consider the underlying sets to obtain a partial order via the subset relation.
For their implementation, Isabelle adds a layer of abstraction, in which a set of classes makes up a sort.
Classes, too, denote a subset of the defined types and have a partial order, just like typeclasses in the Haskell programming language.
The empty sort is the universal sort, containing all types.
Inserting a class into a sort restricts it by intersecting both sets of types.

\begin{quote}
    \begin{isabelle}
        \isacommand{type\_synonym}\ class\ {\isacharequal}\ name \isanewline
        \isacommand{type\_synonym}\ sort\ {\isacharequal}\ class\ set
    \end{isabelle}
\end{quote}

We motivate that Isabelle has two different kinds of variables by presenting the resolution rule, which is involved in almost every interaction a user has with the system.
While it currently resides in the kernel, we have shown how to emulate it using other kernel primitives~\parencite{Madge17}.

\begin{displaymath}
    \prftree[r]{RS}
    {\left\llbracket A_x\,\middle|\, x \in \left[ m \right] \right\rrbracket \Longrightarrow B}
    {\left\llbracket B_x\,\middle|\, x \in \left[ n \right] \right\rrbracket \Longrightarrow C}
    {B\sigma \equiv B_i\sigma}
    {\left( \left\llbracket B_x\,\middle|\, x \in \left[ i - 1 \right] \right\rrbracket
        + \left\llbracket A_x\,\middle|\, x \in \left[ m \right] \right\rrbracket
        + \left\llbracket B_x\,\middle|\, x \in \left( \left[ n \right] \setminus \left[ i \right] \right) \right\rrbracket
        \Longrightarrow C \right) \sigma
    }
\end{displaymath}

It is a generalization of modus ponens (\(A \Longrightarrow (A \Longrightarrow B) \Longrightarrow B\)) enabling the combination of two theorems.
By unifying (\(\sigma\)) the conclusion of the former theorem \(B\) with one of the latter's premises \(B_i\), replacing the assumption \(B_i\) with the former's premises becomes a valid transformation, so successful unification is imperative to applying the rule.
Using RS requires applying the unifier \(\sigma\) to the whole resulting theorem, which may instantiate variables we desire to remain unchanged.
One way of addressing this is limiting the variables that unification may operate on and thus prevent the system from applying the rule.
To express which variables to save from unification, Isabelle offers two kinds:
\isa{Var}s are affected by unification and often referred to as schematic variables, while the unification algorithms treat free variable \isa{Free} like constants.


\begin{quote}
    \begin{isabelle}
        \isacommand{datatype}\ typ\ {\isacharequal}\ TFree\ name sort\isanewline
        \isaindent{\ \ }{\isacharbar}\ TVar\ indexname\ sort\isanewline
        \isaindent{\ \ }{\isacharbar}\ Type\ name\ {\isacharparenleft}typ\ list{\isacharparenright}
    \end{isabelle}
\end{quote}

The final constructor Type is very versatile and takes a name and a list of arguments.
An integer type constant, which comes with an empty list, is one of the several examples to illustrate its use.
For the aforementioned \isa{\(\alpha\) list}, we have the singleton list \isa{[\(\alpha\)]}, and the function \isa{\(\alpha\) \isasymRightarrow\ \(\beta\)} needs \isa{[\(\alpha\), \(\beta\)]}.
We can already see that not all instances of \isa{typ} are necessarily valid, as there is a fixed argument count for each type.
In \autoref{sec:validity}, we will provide rules for both types and terms to ascertain validity.

\section{Terms}

The term language should be rather unsurprising at this point.
Our basic terms are constants, potentially polymorphic, and variables.
The two distinct kinds of variables make a reappearance, but we complement them with bound variables.
They are the counterpart to the name-giving lambda abstractions, or binders, encoded by \isa{Abs}.
Together both allow defining anonymous functions, like the successor function \(\lambda x.\ x + 1\).
The idea is to replace all the bound variable occurrences with the argument applied to the abstraction when reducing a term, which we call \(\beta\)-reduce.
This way, we can obtain a term for the successor of 5: \((\lambda x.\ x + 1)\ 5 \equiv_\beta 5 + 1\).
To encode the function application, we use the \isa{\$}, so \(1 + 2\) becomes \isa{(+) \$ 1 \$ 2}.
While it appears straightforward to let bound and regular variables share a namespace, this approach comes with complications.
When substituting variables by terms, binders around the variable may capture previously unbound variables, as in \(\lambda y. (\lambda x y.\ f\ x\ y)\ y\).
Capturing can be avoided by renaming bound variables, referred to as \(\alpha\)-conversion, but requires additional effort.
Another inconvenience is that \(\alpha\)-equivalent terms, syntactically equal after \(\alpha\)-conversion, can take different shapes: \(\lambda x.\ x \equiv_\alpha \lambda y.\ y\).
Using de Bruijn indices~\parencite{DeBruijn72} instead remedies both of these shortcomings.
Here, the bounds have no lables but refer to their binder through their distance, as measured by binders in between.
While the current Pure implementation still attaches labels to the binders for convenience, they serve no logical purpose and are consequently not part of this formalization.

\begin{quote}
    \begin{isabelle}
        \isacommand{datatype}\ term\ {\isacharequal}\ Const name typ\isanewline
        \isaindent{\ \ }{\isacharbar}\ Free name typ\isanewline
        \isaindent{\ \ }{\isacharbar}\ Var indexname typ\isanewline
        \isaindent{\ \ }{\isacharbar}\ Bound nat\isanewline
        \isaindent{\ \ }{\isacharbar}\ Abs typ term\isanewline
        \isaindent{\ \ }{\isacharbar}\ term \$ term
    \end{isabelle}
\end{quote}

\section{Validity}
\label{sec:validity}

Now that we can write down both types and terms, we want to separate the wheat from the chaff.
To that end, we have to deal with three layers, term, type, and sort.
First, we take care of them separately, and then we check whether the type annotations in a given term are consistent with its structure through type checking.
On all layers, we will ensure that the constants utilized are defined and used according to our expectations.
To that end, we formalize a type signature \(\Sigma\), which encodes those expectations.

\begin{quote}
    \begin{isabelle}
        \isacommand{datatype}\ signature\ {\isacharequal}\ Signature\ \isanewline
        \isaindent{\ \ }(const\_typ\_of{\isacharcolon}\ name \isasymrightharpoonup\ typ)\isanewline
        \isaindent{\ \ }(typ\_arity\_of{\isacharcolon}\ name \isasymrightharpoonup\ nat)\isanewline
        \isaindent{\ \ }(sorts\_of{\isacharcolon}\ algebra)
    \end{isabelle}
\end{quote}

The explicitly partial \isa{('a \isasymrightharpoonup\ 'b \isasymequiv\ 'a {\isasymRightarrow} 'b option)} function \isa{const\_typ\_of} maps term constants to their potentially polymorphic type.
Every type constructor has a fixed arity, which \isa{typ\_arity\_of} returns.
The algebra combines a binary relation (\(\leq\)) over the classes with an extended idea of type arities.
The extension aims to capture the restrictions imposed on the arguments of a type constructor in the context of the \isa{of\_sort :: typ \isasymRightarrow\ sort \isasymRightarrow\ bool} relation, which indicates whether a type belongs to a sort.
We achieve this with a two-leveled map returning a vector of sort requirements.
It has an entry for each argument of a type constructor and only exists for individual type classes.

\begin{quote}
    \begin{isabelle}
        \isacommand{datatype}\ algebra\ {\isacharequal}\ Algebra\ \isanewline
        \isaindent{\ \ }(classes{\isacharcolon}\ class\ rel)\isanewline
        \isaindent{\ \ }(arities{\isacharcolon}\ name \isasymrightharpoonup\ class \isasymrightharpoonup\ sort list)
    \end{isabelle}
\end{quote}

Now that we have our data structures in place, we can define some operations on them and discuss validity criteria.
In the following, we fix the name \isa{cs} for our class relation and assume that it is a partial order and the relation's field to be finite.
Later we will hide that behind the \isa{ordered\_classes} predicate.

\Snippet{ordered_classes}
It follows from our definitions that any sort that exists must be finite because there are only finitely many classes.
Like the relation on classes, \isa{sort\_leq} is also a partial relation.
Using those definitions, we can state the requirements for the validity of arities.
Trivially we expect all sorts used to be well-defined.
Without loss of generality, we require any constructor to have type arities defined for all classes.
The critical property of the arities is that they are coregular, which means that the relation between two classes dictates that of the induced arities.

\begin{quote}
\begin{isabelle}
\isacommand{definition} arities{\isacharunderscore}ok\ arss\ {\isacharequal}\isanewline
{\isacharparenleft}{\isasymforall}ars{\isasymin}ran\ arss{\isachardot}\isanewline
\isaindent{{\isacharparenleft}\ \ \ }{\isacharparenleft}{\isasymforall}c\isactrlsub {\isadigit{1}}{\isasymin}dom\ ars{\isachardot}\isanewline
\isaindent{{\isacharparenleft}\ \ \ {\isacharparenleft}\ \ \ }{\isasymforall}c\isactrlsub {\isadigit{2}}{\isasymin}dom\ ars{\isachardot}\isanewline
\isaindent{{\isacharparenleft}\ \ \ {\isacharparenleft}\ \ \ \ \ \ }class{\isacharunderscore}leq\ c\isactrlsub {\isadigit{1}}\ c\isactrlsub {\isadigit{2}}\ {\isasymlongrightarrow}\isanewline
\isaindent{{\isacharparenleft}\ \ \ {\isacharparenleft}\ \ \ \ \ \ }list{\isacharunderscore}all{\isadigit{2}}\ sort{\isacharunderscore}leq\ {\isacharparenleft}the\ {\isacharparenleft}ars\ c\isactrlsub {\isadigit{1}}{\isacharparenright}{\isacharparenright}\ {\isacharparenleft}the\ {\isacharparenleft}ars\ c\isactrlsub {\isadigit{2}}{\isacharparenright}{\isacharparenright}{\isacharparenright}\ {\isasymand}\isanewline
\isaindent{{\isacharparenleft}\ \ \ }dom\ ars\ {\isacharequal}\ Field\ cs\ {\isasymand}\ {\isacharparenleft}{\isasymforall}sv{\isasymin}ran\ ars{\isachardot}\ list{\isacharunderscore}all\ sort{\isacharunderscore}ex\ sv{\isacharparenright}{\isacharparenright}%
\end{isabelle}
\end{quote}

Bringing everything together, we can now state what it means for an algebra to be well-formed.

\Snippet{algebra_ok}

With a solid understanding of the sort algebra, we can now define the previously teased \isa{of\_sort} predicate.
As the type arities only have information on how the constructors behave on the class level, we lift them to sorts using \isa{mg\_domain}.
That function looks up the constituting classes' arities and intersects all those argument constraints by folding the componentwise intersection of two sort lists over them.
Fortunately, the variable cases can simply delegate to the \isa{sort\_leq} relation.

\Snippet{of_sort}

Before going a level up and defining a criterion for signatures, we explain what it means for terms, types, and sorts to adhere to them.
All predicates for them have a prime appended to denote their dependence on \(\Sigma\).
Starting easy, \isa{sort\_ok'} only requires unpacking the signature and calling \isa{sort\_ex}.

\Snippet{typ_ok'}

When considering types, we want to ensure that the constructors' arities match the number of arguments found in the type and check all the sorts we encounter.
In the case of terms, the focus again is on the constants.
They can be polymorphic, complicating things.
Thus, while perfectly valid, the type annotation in the term might not be equal to the most general one in \(\Sigma\), so we need an order on types regarding generality.
Our approach is to use matching, which is unification restricted to one side.
If we can find such a unifier, the instantiated term was either equally general or more general than the fixed one.
To do that, we need a notion of type instantiation.
Preserving sort constraints and validity of the whole type is essential, so we check that the new types belong to the sort of the variable they are replacing and are valid themselves.

\Snippet{ty_is_instance}

Using \isa{ty\_is\_instance}, we can check that a given constant is valid with regards to \(\Sigma\) by looking for a unifier that matches the polymorphic type to the one found in the constant.
Additionally, we delegate to the type level.

\Snippet{term_ok'}

What we haven't covered yet, is regular type checking.
It is about making sure that all bounds are bound and function application respects the types.
To that end, we reserve the name \isa{fun} for the function type, assign it the arity two, and use the abbreviation \(\alpha\ \isasymrightarrow\ \beta \equiv\) \isa{Type {\isacharprime}{\isacharprime}fun{\isacharprime}{\isacharprime} [\(\alpha\),\(\beta\)]}.
As is often the case with De Bruijn indexed terms, we need to keep track of surrounding binders when descending into a term, so we put them in an accumulator.
Type checking can fail, so we return an option type and use the option monad for succinctness.

\Snippet{typ_of}

Using all those definitions, we can finally state what makes a valid signature.
We require the types of the declared constants to be valid, the function type's arity to be correct, and the sort algebra to be valid.
Furthermore, we connect the type arities, with the sort algebra, by requiring a shared domain and matching argument counts.

\begin{quote}
\begin{isabelle}%
\isacommand{definition} signature{\isacharunderscore}ok\ {\isacharparenleft}Signature\ const{\isacharunderscore}typ\ typ{\isacharunderscore}arity\ sorts{\isacharparenright}\ {\isacharequal}\isanewline
\ {\isacharparenleft}{\isasymforall}ty{\isasymin}ran\ const{\isacharunderscore}typ{\isachardot}\ {\isacharparenleft}typ{\isacharunderscore}ok{\isacharprime}\ {\isacharparenleft}Signature\ const{\isacharunderscore}typ\ typ{\isacharunderscore}arity\ sorts{\isacharparenright}{\isacharparenright}\ {\isasymand}\isanewline
\isaindent{\ }typ{\isacharunderscore}arity\ {\isacharprime}{\isacharprime}fun{\isacharprime}{\isacharprime}\ {\isacharequal}\ Some\ {\isadigit{2}}\ {\isasymand}\isanewline
\isaindent{\ }algebra{\isacharunderscore}ok\ sorts\ {\isasymand}\isanewline
\isaindent{\ }dom\ typ{\isacharunderscore}arity\ {\isacharequal}\ dom\ {\isacharparenleft}arities\ sorts{\isacharparenright}\ {\isasymand}\isanewline
\isaindent{\ }{\isacharparenleft}{\isasymforall}n{\isasymin}dom\ typ{\isacharunderscore}arity{\isachardot}\isanewline
\isaindent{\ {\isacharparenleft}\ \ \ }{\isasymforall}os{\isasymin}ran\ {\isacharparenleft}the\ {\isacharparenleft}arities\ sorts\ n{\isacharparenright}{\isacharparenright}{\isachardot}\ the\ {\isacharparenleft}typ{\isacharunderscore}arity\ n{\isacharparenright}\ {\isacharequal}\ {\isacharbar}os{\isacharbar}{\isacharparenright}{\isacharparenright}%
\end{isabelle}
\end{quote}

\section{Term Operations}

Before putting the terms to their ultimate use, we will define type instantiation and the duals \(\beta\)-application and abstraction.
For this section's function definitions, we omit the cases where the term remains unchanged, irrespective of the other arguments.
Lifting \isa{ty\_instT} to the term level is straightforward with \isa{map\_types}, which maps any function over all type annotations of a term.

\Snippet{instT}

Later, we will need to abstract over a subterm, i.e., replacing all occurrences of the subterm with a bound pointing to the top level.
This requires us to keep track of the current distance when descending into the term, so we need an accumulator.
All that is left then is attaching a new binder with the subterm's type with \isa{mk\_Abstraction v t \isasymequiv\ Abs (the (typ\_of v)) (abstract\_over v t)}.

\Snippet{abstract_over}

The story is more involved for \(\beta\)-application \isa{(\isasymbullet)} because we need to handle the De Bruijn indices correctly.
When placing a term with unbound bounds into a bigger context, they can become bound by the additional binders.
Although type checking rejects such a term, we avoid requiring type correctness by means of incrementing those dangling bounds by the depth of the term's new position.

\Snippet{incr_boundvars}

Substituting bound variables necessitates removing a binder.
After that, \isa{subst\_bound} replaces those bound linked to the binder previously on the top level.
We achieve this by visiting all bound variables and checking their binders.
If they are otherwise bound in the term, we ignore them.
In case the previously outermost binder binds them, we replace them with the new term while dealing with its dangling bounds.
Were they unbound in the first place, we decrement their index by one, to account for the removal of the outermost binder.

\Snippet{subst_bound}

Now we are ready to define \isa{(\isasymbullet)}.
We connect the two terms with the term application, but if we can \(\beta\)-contract the resulting term on the top-level, we do it.

\Snippet{betapply}