\chapter{Related Work}\label{chapter:related}

Artifacts of the work on proof calculi exit going back to the old Greeks~\parencite{Lukasiewicz1951}.
The logical underpinnings of theorem provers, in particular, have also been extensively studied to different degrees of rigor.
Informal descriptions exist for both HOL~\parencite{Gordon94} and CoC~\parencite{Huet88}, the logic underlying the Coq theorem prover.
Different authors have formalized more concrete specifications of proof calculi within provers, either the systems themselves or different ones.
John Harrison, who created the HOL Light prover, formalized its logic and self-verified his implementation against it~\parencite{Harrison06}.
Kumar et al.\ ported those results to the HOL4 prover and integrated it into the CakeML ecosystem~\parencite{Kumar2016}, resulting in a verification down to machine language.\\

Isabelle's foundations somewhat differ from those of other systems of the LCF family.
One of those ways is the support for overloaded constants.
This ad-hoc overloading is challenging to get right, as Kun{\v c}ar and Popescu have shown~\parencite{Kuncar19}.
Pohjola and Gengelbach have recently adapted~\parencite{Kumar2016} to support overloaded constants using those ideas.